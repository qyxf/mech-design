\documentclass[12pt,a4paper]{article}
\usepackage[margin=1in]{geometry}

\usepackage[heading]{ctex}
\usepackage{amsmath}
\usepackage{unicode-math}
\setCJKmainfont{思源宋体}
\setCJKsansfont{思源黑体}
\setmainfont{Cambria}
\setsansfont{Calibri}
\setmathfont{Cambria Math}

\usepackage{fancyhdr}
\usepackage{xcolor}
\lhead{}\rhead{}
\chead{\textsc{Outline on Fundamentals of Mechanical Design}}
\pagestyle{fancy}
\setcounter{secnumdepth}{0}
\setlength{\headheight}{15pt}

\usepackage{siunitx}
\usepackage{fontawesome5}

\newcommand{\tightlist}{\setlength{\parskip}{0pt}\setlength{\itemsep}{0pt}}
\newcommand{\hint}[1]{\textsf{(#1)}}
\newcommand{\minor}[1]{{\color{gray} #1}}
\newcommand{\then}{$\to$}
\newcommand{\equv}{$\Leftrightarrow$}
\newcommand{\beginday}{2019 年 6 月 30 日}
\renewcommand{\emph}[1]{\faIcon[regular]{lightbulb}\ \textbf{#1}}

\DeclareMathOperator{\inv}{inv}

\title{《机械设计基础》考点提纲}
\date{\beginday\ -- \today}
\author{黑山雁}

\begin{document}
\maketitle

\section{绪论与概述}
\subsection{基本概念}
\begin{itemize}\tightlist
    \item 机器:执行机械运动的装置\hint{人为实体组合、确定相对运动、功能转换或做有用功}。
    \item 机构:只有运动变换,没有能量转换;机器的运动部分\hint{剔除了与运动无关的因素}。
    \item 构件:运动的单元体;零件:加工制造的单元体\hint{注意区分}。
    \item \minor{机械:机器与机构的总称。}
\end{itemize}

\subsection{设计原则}
\begin{itemize}\tightlist
    \item 传统设计方法:依靠基础物理与经验汇总;现代设计方法:计算机辅助、优化、可靠性
    设计。
    \item \hint{零件}失效:不能在预定条件\hint{载荷}、预定期限\hint{寿命}下正常工作;工作能力:抵抗失效的能力。
    \item \minor{功能分析:总功能\then 功能元\then 功能载体\hint{机构、零件};原理分析
    :形态学矩阵,择优选取。}
    \item 机械设计基本要求:
    \begin{enumerate}\tightlist
        \item 功能要求:实现预定功能,不失效。
        \item 可靠性要求:减少零件数目。\emph{零件越多,可靠度越低。}
        \item 经济性要求:能用便宜的,就不用贵的。
        \item \minor{操作方便和安全要求。}
        \item \minor{造型、色彩要求。}
    \end{enumerate}
    \item 零件设计一般程序\hint{之后各章遵循的逻辑}:
    \begin{enumerate}\tightlist
        \item 初选结构形式、选择零件类型\hint{读型号,分析需求,结合零件特点查表选型}。
        \item 计算作用在零件上的载荷\hint{受力、应力分析}。
        \item 选择零件材料及热处理方式\hint{「工程材料基础」、「金工实习」}。
        \item 零件工作能力设计\hint{强度、刚度、疲劳、振动、热平衡}。
        \item 零件结构设计\hint{尺寸、形状}。
        \item 校核计算\hint{确定一个,核另一个}。
        \item \minor{绘制零件工作图、编写设计计算说明书}。
    \end{enumerate}
    \item 三种设计:开发性设计\hint{创新}、适应性设计\hint{改进}、变型设计\hint{改参数}。
\end{itemize}

\section{运动设计基础}
\subsection{构件与运动副}
\begin{itemize}\tightlist
    \item 机构的组成:构件\hint{运动单元体;用一条线表示}、运动副\hint{两部分保持接触、
    形成可动连接;用不同符号表示}。
    \item \emph{两表面不接触,就没有运动副。}
    \item 运动副分类:
    \begin{itemize}\tightlist
        \item 按相对运动形式:平面运动副\hint{转动副、移动副、平面滚滑副},空间运动副
        \minor{\hint{螺旋副、球面副、圆柱副……}}。
        \item 按两构件接触情况:低副\hint{面接触}、高副\hint{线接触、点接触}。
        \item 按锁合方式:形锁合、力锁合。
    \end{itemize}
    \item 运动链:构件由运动副连接而成的系统;分类:开式链、闭式链
    \hint{首尾相连;单闭环链、双闭环链……}。
    \item 运动链成为机构的条件:有且仅有一个机架;具有确定运动规律\hint{主动件数目等于
    自由度}。
\end{itemize}

\subsection{机构的运动简图与运动特性}
\begin{itemize}\tightlist
    \item 机构运动简图:简化、抽象,与实际机构具有相当的运动特性。
    \item 机构简图绘制步骤\hint{从实物到模型}:分清构件\then 判断运动副类型\then 选择
    视图、主动件位置,测量运动尺寸\then 绘制。
    \item 比例尺标注示例:$\mu_l=\SI{0.001}{m/mm}$\hint{$1:1$也得这么写}。
    \item \emph{绘图四要素:比例尺;实际尺寸;主动件;机架。\hint{不要漏项}}
    \item \minor{不便在平面视图内绘图:轴测图、展开图、增加辅助视图平面;机动示意图
    \hint{不严格按比例绘制}。}
    \item 构件自由度:确定构件运动所需独立参数个数\hint{无约束平面运动:$f=3$}。
    \item 机构自由度:机构具有的独立运动数目。
    \begin{itemize}\tightlist
        \item 平面低副约束$2$(空间中为 $5$)个自由度,平面高副约束$1$(空间中为 $4$)个自由度。
        \item 平面机构自由度公式:$F=3n-2P_5-P_4$,其中$n=N-1$扣去机架。
    \end{itemize}
    \item 自由度的指导意义:设主动件数为$A$,则:
    \begin{itemize}\tightlist
        \item $A>F$:机构被破坏\hint{硬掰}。
        \item $A<F$:机构无确定运动\hint{乱甩}。
        \item $A=F$:机构具有确定运动。
    \end{itemize}
    \item 自由度计算时的特殊情况:
    \begin{itemize}\tightlist
        \item 复合铰链:$K$个构件共一转动副,等效为$K-1$个转动副。
        \item 局部自由度:常见为滚子圆,直接「焊接」。
        \item 虚约束:不起作用的运动副,直接去掉。\emph{共性:相对位置较为特殊,稍偏离就不能形成运动副,或过约束。}
        \begin{enumerate}\tightlist
            \item 轨迹重合:平行四边形机构。
            \item 多处接触:刚好两边都卡紧。
            \item 对称部分:行星机构对称轮。
        \end{enumerate}
        \item 公共约束:常见为平面楔块机构,公式改为$F=2n-P_5$。
    \end{itemize}
    \item 速度瞬心:两构件上瞬时绝对速度相等的某点,可变化。
    \begin{itemize}\tightlist
        \item 绝对瞬心:两构件之一为机架\hint{瞬时静止点,如无滑滚动接触点}。
        \item 相对瞬心:两构件均运动。
    \end{itemize}
    \item 机构中瞬心总数目:$K=\binom{N}{2}=\frac{N(N-1)}2$\hint{$N$含机架}。
    \item 基本运动副瞬心:
    \begin{itemize}\tightlist
        \item 转动副:转动中心。
        \item 移动副:垂直导路方向的无穷远处\hint{需记为$P_{12}^\infty$}。
        \item 平面滚滑副:在过接触点的公法线上,待定。
    \end{itemize}
    \item 三心定理:互作平面运动的三构件之三瞬心在同一直线上\hint{固定一构件,则两运动
    构件之瞬心速度受两个转动方向调制}。
    \item 利用三心定理求瞬心:找到包含待求瞬心所在的两组三心带,利用三心定理获得该瞬心所在的两条直线,其交点即为瞬心之所在。
    \item 用瞬心分析速度:利用瞬心处速度相等条件列等式,求解未知量。\emph{标注方向。}
    \item \emph{计算速度时,一定要将比例尺代入,不可省略。}
    \item 计算规范:理论依据\hint{公式}\then 原始数据\hint{代值}\then 计算结果。
\end{itemize}

\section{工作能力与结构设计基础}
\subsection{工作能力分析}
\begin{itemize}\tightlist
    \item 载荷:机械零件在工作中所受的外力;分类:静载荷;动载荷\hint{周期载荷:对称、
    脉动、非对称;非周期载荷:准周期、瞬变;随机载荷}。
    \item \minor{载荷的确定:类比法\hint{经验}、计算法\hint{公式}、实测法\hint{如电测}。}
    \item 应力分类:静应力与动应力、体积应力\hint{拉压弯扭}与表面应力\hint{挤压、接触}、
    温度应力与装配应力。
    \item \minor{两圆柱体接触应力:$\sigma_\text{Hmax}=\sqrt{\frac F{\pi b}\cdot
    \frac{1/\rho_\text{v}}{(1-\mu_1^2)/E_1+(1-\mu_2^2)/E_2}}$,其中$\rho_\text{v}
    =\left(\frac1{\rho_1}\pm\frac1{\rho_2}\right)^{-1}$ 为当量曲率半径\hint{正外负内}。}
    \item 机械零件失效形式:断裂、表面破坏\hint{压溃、磨损、点蚀、胶合}、过量变形
    \hint{刚度不足}、其他失效\hint{带打滑、螺纹松动、油膜破裂}。
    \item 工作能力的设计计算准则:强度准则、刚度准则、耐磨性准则\hint{滑动轴承中采用}、
    振动稳定准则\hint{轴中采用}、可靠性准则\hint{滚动轴承采用}。
    \item 材料选用:
    \begin{enumerate}\tightlist
        \item 应满足零件的使用要求\hint{要可靠}。
        \item 应满足零件的工艺要求\hint{要扛得住加工}。
        \item 应满足经济性要求\hint{不要太贵}。
    \end{enumerate}
    \item 强度分类:体积强度与表面强度、静应力强度与动应力强度\hint{疲劳强度}。
    \item 强度判断方法:许用应力法\hint{$\sigma\leq[\sigma]$}、许用安全系数法\hint%
    {$S_\sigma=\frac{\sigma_b}{\sigma}\geq[S_\sigma]$}。
    \item \minor{强度计算原则:简单情况直接计算;复杂情况下,塑形材料采用第三强度理论,
    脆性材料采用第一强度理论计算当量应力;接触强度与挤压强度单独计算。}
    \item \minor{其他知识点:疲劳强度\hint{可参见材料力学}、刚度计算\hint{具体考量}。}
    \item \minor{摩擦:干摩擦、边界摩擦、流体摩擦、混合摩擦。}
    \item 磨损:磨合磨损阶段\hint{急}\then 稳定磨损阶段\hint{缓}\then 剧烈磨损阶段
    \hint{急};设计原则:缩短磨合阶段,延长稳定阶段,推迟剧烈阶段。
    \item 磨损分类:粘着磨损\hint{冷焊}、磨粒磨损、疲劳磨损\minor{、冲蚀磨损、腐蚀磨损}。
    \item \emph{有磨损就有润滑,两者必须同时考虑。}
\end{itemize}

\subsection{结构设计}
\begin{itemize}\tightlist
    \item 结构设计的意义:为计算提供依据;确定结构尺寸。
    \item 结构设计方法:变换
    \begin{itemize}\tightlist
        \item 变换结构形态:形状、位置、数目、尺寸;
        \item 变换零件关系:运动形式\hint{滑动与滚动}、连接方式。
    \end{itemize}
    \item 结构设计的考虑因素:
    \begin{enumerate}\tightlist
        \item \minor{合理分配功能:争取一个功能载体实现多种功能。}
        \item 提高刚度强度:等强度结构、改良截面形状\hint{工字梁}、改善受力状况\hint{
        分载、均化}、减少应力集中\hint{增大过渡曲率半径、卸载结构}、强化与预紧。
        \item 提高耐磨性\minor{:降低压强、脱离接触、均匀磨损、材料组合}。
        \item 改善工艺性\minor{:考虑生产条件与批量、力求简单、易于装配}。
    \end{enumerate}
\end{itemize}

\section{平面连杆机构}
\begin{itemize}\tightlist
    \item 定义:若干刚性构件、用低副连接、各构件做平面运动。
    \item 优点:实现多种运动形式\hint{整周转动、摆动、移动}转换;实现复杂运动轨迹;磨损
    轻;形状简单、制造方便。
    \item 缺点:只能近似实现运动要求;设计复杂;惯性力难以平衡,振动与动载荷明显,只能
    用于低速。
\end{itemize}

\subsection{平面连杆机构大赏}
\begin{itemize}\tightlist
    \item 基本形式:两杆机构\hint{不能转换运动}\then 三杆机构\hint{无自由度}\then
    四杆机构\hint{基本形式}
    \item 铰链四杆机构:最基本的平面四杆机构,全由转动副连接。
    \begin{itemize}\tightlist
        \item 结构拆分:机架\hint{固定}、连架杆\hint{与机架相连}、曲柄\hint{可整周转动的连架杆}、
        摇杆\hint{只能摆动的连架杆}、连杆\hint{与机架相对}。
        \item 曲柄摇杆机构:等速整圈转动\equv 变速往复摆动;实例:颚式破碎机、缝纫机
        脚踏板。
        \item 双曲柄机构:等速转动\then 变速转动\hint{可以实现较大的力与急回特性};
        特例:平行四边形机构\hint{火车}、反平行四边形机构\hint{门窗启闭装置}。
        \item 双摇杆机构:匀速摆动\then 变速摆动;特例:等腰梯形机构\hint{前轮转向机构}。
    \end{itemize}
    \item 连杆运动轨迹:连杆曲线\hint{代数曲线}。
    \item \emph{连架杆的运动轨迹无意义,都是圆弧。}
    \item 铰链四杆机构的演化:
    \begin{itemize}\tightlist
        \item 变换构件形状:
        \begin{enumerate}\tightlist
            \item 曲柄滑块机构:转动\then 平动\footnote{一般不反过来,因为会卡在死点。
            },分对心与偏置;实例:内燃机。
            \item 双滑块机构:平动\then 平动;实例:椭圆规。
            \item \minor{余弦机构:常用于计算装置中。}
        \end{enumerate}
        \item 变更机架位置:
        \begin{enumerate}\tightlist
            \item 曲柄滑块机构\then 导杆机构\hint{机架:原曲柄}\minor{\then 摇块机构
            \hint{机架:原连杆}\then 定块机构\hint{机架:原滑块}}。
            \item \minor{双滑块机构\then 正弦机构\hint{机架:原滑块}\then 双转块机构
            \hint{机架:原连杆}。}
        \end{enumerate}
        \item 扩大转动副:曲柄滑块机构\then 偏心轮机构\hint{可提高强度与刚度}。
    \end{itemize}
\end{itemize}

\subsection{四杆机构分析与设计}
\begin{itemize}\tightlist
    \item \emph{原动机\hint{如电机}是整周转动,故希望四杆机构中存在曲柄。}
    \item 四杆机构曲柄存在条件:
    \begin{enumerate}\tightlist
        \item 最短杆长度$+$最长杆长度$\leq$其余两杆长度之和\hint{长杆不能太长}。
        \item 连架杆与机架之中必有一个是最短杆\hint{最短杆不是连杆;不做梯形机构}。
    \end{enumerate}
    \item 满足第一条件后,取连架杆为最短杆得曲柄摇杆机构,取机架为最短杆得双曲柄机构,
    否则得双摇杆机构。
    \item 极限位置:往复运动\hint{摆动、移动}构件的两个极端位置;摆角$\psi$\hint{对
    摇杆}与极位夹角$\theta$\hint{对曲柄,总是锐角}。
    \item 行程速比系数:$K=\frac{\omega_R}{\omega_W}=\frac{\SI{180}\degree+\theta}%
    {\SI{180}\degree-\theta}>1$。
    \item 设计极位夹角:$\theta=\SI{180}\degree\cdot\frac{K-1}{K+1}<\SI{180}\degree$。
    \item 压力角$\alpha$\hint{对从动件}:受力点处,载荷\hint{外力}方向与速度方向夹角;
    传动角$\gamma$\hint{仅对连杆机构}:连杆与从动件所夹锐角。
    \begin{itemize}\tightlist
        \item 压力角越小,机构传力效果越好。 
        \item 传动角与压力角换算:$\gamma=\SI{90}\degree-\alpha$。
        \item 传动角测量:连杆与从动件夹角为$\gamma'$,则当$\gamma'\leq\SI{90}\degree$
        时$\gamma=\gamma'$,$\gamma'>\SI{90}\degree$时$\gamma=\SI{180}\degree-
        \gamma'$。
    \end{itemize}
    \item 最小传动角$\gamma_\text{min}$:衡量机构传力效果下限;图解、计算按
    \[\gamma_\text{min}=\min\{\gamma'_\text{min},\SI{180}\degree-\gamma'_\text{max}\}.\]
    \item 最小传动角图解:曲柄主动的机构,其 $\gamma_\text{min}$ 必出现于曲柄、机架共线处。
    \item \emph{在学习与备考过程中,用作图测量法(而非余弦定理)求解最小传动角。}
    \item 死点:$\gamma=0$的位置;出现于摇杆作主动件时,连杆与曲柄成直线;出现机构自锁
    或反转现象。
    \item \minor{死点的克服:利用构件惯性或其他机构辅助;死点的利用:分合闸、机床上的
    夹紧装置。}
    \item 运动连续性:检验机构是否能通过运动\hint{而非在静止时}到达指定位置。
    \item 平面连杆机构运动设计命题:
    \begin{itemize}\tightlist
        \item 刚体引导问题:要求实现几个给定位置。
        \item 要求实现给定的运动规律\hint{达成给定的急回特性$K$}。
        \item \minor{要求实现给定轨迹\hint{连杆曲线}。}
        \item 附加要求:存在曲柄、运动连续性、$\gamma_\text{min}\geq[\gamma]$\minor%
        {、尺寸符合要求}。
    \end{itemize}
    \item 设计方法:图解法\minor{、实验法\hint{图谱或模型}、解析法}。
    \item 图解常用技巧:作圆弧、作中垂线、利用圆心角/圆周角。
\end{itemize}

\section{凸轮机构}
\begin{itemize}\tightlist
    \item 凸轮机构的组成:凸轮、从动件、机架;凸轮机构为高副机构。
    \item 优点:结构简单紧凑、运动可靠、可实现复杂运动规律;缺点:高副机构,易于磨损,
    一般不用于传力,只用于运动控制。
\end{itemize}

\subsection{基本概念与运动规律}
\begin{itemize}\tightlist
    \item 凸轮的类型:凸轮与从动件相互作平行平面运动,则为平面凸轮,否则为空间凸轮。
    \begin{itemize}\tightlist
        \item 平面凸轮:盘形凸轮\hint{凸轮作整圈转动}、移动凸轮\hint{凸轮作往复移动}。
        \item \minor{空间凸轮:圆柱凸轮、圆锥凸轮、弧面凸轮、端面凸轮。}
    \end{itemize}
    \item 从动件的类型:尖端\hint{理论从动件,磨损大}、滚子\hint{磨损小,应用最广}、
    平底\hint{作用力方向稳定,磨损小,但不能用于内凹轮廓}\minor{、曲面\hint{介于尖端与
    平底之间}}。
    \item 凸轮机构命名:安装方式\hint{对心、偏置} $+$ 从动件形状\hint{尖端、滚子、
    平底、曲面} $+$ 从动件类型\hint{移动从动件、摆动从动件} $+$ 凸轮类型\hint{盘形凸轮、
    移动凸轮、各类空间凸轮……} $+$ 「机构」。
    \item 例——最常见的凸轮:对心滚子移动从动件盘形凸轮机构。
    \item 锁合装置的类型:力锁合\hint{依靠重力等,最常见}、形锁合\hint{沟槽、等宽、等
    径、共轭}。
    \item 理论廓线:从动件上某一「特征点」绘出的轨迹。
    \begin{itemize}\tightlist
        \item 尖端从动件:即尖端点在凸轮平面上绘出的轨迹。
        \item 滚子从动件:即滚子圆中心绘出的轨迹。
        \item \minor{平底从动件:为凸轮中心向平底所作的垂足之运动轨迹\hint{若平底从动件对心安装,
        则理论廓线即为平底与凸轮接触点绘出的轨迹。}}
    \end{itemize}
    \item 实际廓线/工作廓线:即凸轮的实际轮廓,与理论廓线有一定关系:
    \begin{itemize}\tightlist
        \item 尖端从动件:与理论廓线一致。
        \item 滚子从动件:为理论廓线按滚子圆半径「内缩」而成的等距曲线。
        \item \minor{平底从动件:为从动件的平底在不同位置形成的包络线。}
    \end{itemize}
    \item 其他概念:基圆\hint{以凸轮轴心为圆心,以理论廓线最小向径为半径$r_b$}、行程$h$/
    $\psi_\text{max}$\hint{从动件的最大位移}、推程$\varphi_t$\then 远休止$\varphi_s$
    \then 回程$\varphi_h$\then 近休止$\varphi'_s$。
    \item 从动件的运动规律:位移 $s$、速度 $v$ 与加速度 $a$ 关于凸轮转角 $\varphi=\omega
    t$ 的变化规律;推导:根据要求列运动方程,并代入 $h$、$\varphi_t$ 等边界条件。
    \item 从动件基本运动规律:
    \begin{enumerate}\tightlist
        \item 等速运动规律:$s=\frac h{\varphi_t}\varphi$,有刚性冲击\hint{始末两点处
        加速度趋于无穷大},只适用于低速场合。
        \item 等加速度运动规律:$s=\frac{2h}{\varphi_t^2}\varphi^2$\hint{前半行程和后半
        行程方向相反},有柔性冲击\hint{交接处加速度有突变},只适用于中低速场合。
        \item 余弦加速度运动规律:$s=\frac h2\left(1-\cos\frac\pi{\varphi_t}\varphi
        \right)$,有柔性冲击,但可以通过「升--降--升」的连接变为连续运动。
        \item 正弦加速度运动规律:$s=h\left(\frac\varphi{\varphi_t}-\frac1{2\pi}\sin
        \frac{2\pi}{\varphi_t}\varphi\right)$,全程连续,可用于高速场合。
    \end{enumerate}
    \item 实际运动中,可用高次多项式运动规律,或组合基本运动规律。
    \item 运动规律设计原则:
    \begin{enumerate}\tightlist
        \item 对大质量从动件,降低从动件最大速度 $v_\text{max}$\hint{以减小动量};
        \item 高速场合下,降低从动件最大加速度 $a_\text{max}$\hint{以减小推力};
        \item 避免各类冲击。
    \end{enumerate}
\end{itemize}

\subsection{轮廓设计}
\begin{itemize}\tightlist
    \item 轮廓设计的依据:从动件运动规律、凸轮基圆半径。
    \item \minor{两类方法:图解法与解析法。}
    \item 图解法的基本原理:「反转法」原理\hint{附加反向加速度,凸轮静止而从动件反转}。
    \item 图解法基本步骤:绘制基圆\then 等分行程曲线\then 量取各角度行程\then 反转速度
    方向上按行程绘制轨迹点\then 按从动件类型完成曲线\hint{尖端:直接连接;滚子:作若干
    滚子圆后内包络曲线;平底:轨迹点处作切向线包络曲线}。
    \item \emph{休止段直接作圆弧即可,不要无谓地标注滚子圆。}
    \item 偏心从动件:在基圆内绘制一个同心圆,作理论廓线时的行程按偏心圆的切线作出。
    \item \minor{摆动从动件:分别作按$r_b$作基圆、按转动中心距$l_{OC}$作中心圆,在中
    心圆上划分点,在划分点上按摆杆长$l$作弧分基圆,进一步偏转得理论廓线上的点。}
    \item 设计基本尺寸:基圆半径$r_b$、滚子半径$r_k$\minor{、平底尺寸}、偏距$e$\minor{
    、摆杆长$l$、中心距$l_{OC}$}。
    \item $r_k$ 的选取:设理论廓线与实际廓线的最小曲率半径为$\rho_\text{min}$、
    $\rho_\text{cmin}$,则它们之间的关系为
    \[ \rho_\text{cmin}=\rho_\text{min}-r_k. \]
    当 $r_k\geq\rho_\text{min}$,实际廓线将不能满足要求,导致出现磨损或运动失真。一般
    要求$r_k\leq0.8\rho_\text{min}$且$\rho_\text{cmin}>1\sim5\text{mm}$,否则应
    增大 $r_b$。
    \item 压力角校核:凸轮从动件间压力角由曲率半径的变化导致,一般在廓线较陡处校核
    $\alpha_\text{max}\leq[\alpha]$\hint{移动从动件取 $[\alpha]=\SI{30}\degree$,
    摆动从动件取 $[\alpha]=\SI{35}\degree\sim\SI{45}\degree$,回程时压力角几乎无要求}。
    \item 压力角不满足要求时,可增大 $r_b$ 或适当偏置,再重新设计。
    \item 基圆半径选择:保证压力角和滚子圆达标的前提下,适当取小可满足紧凑的要求,适当取
    大可改善性能、提高可靠性。
\end{itemize}

\section{齿轮传动}
\begin{itemize}\tightlist
    \item 特点:既可传动又可传力,传动比准确,传动效率高,工作可靠。缺陷:制造成本高。
    \item 齿轮传动分类:
    \begin{itemize}\tightlist
        \item 按轴的布置:平行轴\hint{圆柱齿轮、非圆齿轮}、相交轴\hint{锥齿轮}、交错轴
        \hint{螺旋齿轮、蜗杆等}。
        \item 按轮齿的齿向:直齿轮、斜齿轮、人字齿轮。
        \item 按工作条件:闭式齿轮\hint{用齿轮箱密封}、开式齿轮。
        \item 按齿廓曲线:渐开线、圆弧、摆线……
        \item 按齿面硬度:软齿面\hint{$\leq350\,\text{HBS}$}、硬齿面\hint{$>350
        \,\text{HBS}$}。
    \end{itemize}
\end{itemize}

\subsection{齿轮与渐开线}
\begin{itemize}\tightlist
    \item 齿廓啮合基本定律:两齿轮的传动比,等于两齿轮连心线被啮合处公法线所分的两段长度
    之反比。
    \item 满足齿廓啮合基本定律的一对齿廓称作共轭齿廓,其传动比可以恒定。
    \item 渐开线:一直线\hint{发生线}沿一圆周\hint{基圆}作纯滚动时,直线上任意一点 $K$ 的轨迹。
    \item 渐开线性质:
    \begin{enumerate}\tightlist
        \item 发生线上滚过长度 $NK$ 等于基圆上滚过的弧长 $\overline{AN}$\hint{纯滚动}。
        \item 发生线为渐开线的法线;渐开线上任意一点处法线必与基圆相切。
        \item 因 $KN$ 为渐开线的曲率半径,故渐开线上离基圆越远处,曲率半径越小,曲率
        越大。
        \item 相同的展角下,基圆半径越大,曲率半径越大。
        \item 基圆内无渐开线。
    \end{enumerate}
    \item 渐开线上基本关系:$\tan\alpha_K=\alpha_K+\theta_K$,其中 $\alpha_K$ 为动
    点绕基圆中心的速度与发生线夹角,$\theta_K$ 为展角。
    \item 渐开线方程:$\begin{cases}\theta_K&=\tan\alpha_K-\alpha_K=\inv\alpha_K\\
    r_K&=\frac{r_b}{\cos\alpha_K}\end{cases}$。
    \item 渐开线齿廓的啮合特性:
    \begin{enumerate}\tightlist
        \item 保证瞬时传动比恒定\hint{啮合点处的公法线同时为两确定基圆的内公切线};
        \item 啮合线与啮合角恒定\hint{啮合线即内公切线只有一条,始终确定,啮合角因而
        确定},传动平稳;
        \item 齿轮具有可分性\hint{传动比与基圆半径成反比},传动比不受安装影响\hint{不
        受节圆半径变化影响}。
    \end{enumerate}
\end{itemize}

\subsection{齿轮的基本参数与安装}
\begin{itemize}\tightlist
    \item 齿轮各部分参数:
    \begin{itemize}\tightlist
        \item 齿数 $z$、齿宽 $b$;
        \item 齿顶圆 $d_a$、齿根圆 $d_f$;
        \item 任意圆周 $r_k$ 上:齿槽宽 $e_K$、齿厚 $s_K$、齿距 $p_K=e_K+s_K$;
        \item 分度圆 $d$:$e$、$s$、$p=e+s$;
        \item 模数 $m$:$p=m\pi$,$d=mz$,是标准化的若干有理数。
        \item 压力角\hint{分度圆上}:$\cos\alpha=\frac{r_b}r$\hint{渐开线方程},我
        国规定为 \SI{20}\degree。
        \item 齿顶高 $h_a$,齿根高 $h_f$,齿顶高系数 $h^\star_a$ 满足 $h_a=h^\star_a
        m$,顶隙系数 $c^\star$ 满足 $h_f=(h^\star_a+c^\star)m$。齿全高 $h=h_a+h_f$。
        已标准化为:正常齿制 $h^\star_a=1$、$c^\star=0.25$;短齿制 $h^\star_a=0.8$、
        $c^\star=0.3$。  
    \end{itemize}
    \item 五个基本参数:$z$、$m$、$\alpha$、$h^\star_a$、$c^\star$;标准齿轮:$m$、
    $\alpha$、$h^\star_a$、$c^\star$ 采用标准值且 $s=e$ \hint{分度圆等分齿槽与齿厚}。
    \item 正确啮合条件\hint{运动时不卡住也不中断}:两齿轮的「空隙」应相当,由渐开线性质
    推得 $p_{b1}=p_{b2}$,进而要求两齿轮模数与压力角都相等\hint{两齿轮「型号」相同}。
    \item 极限啮合线与实际啮合线:前者为齿廓公切线的两个切点\hint{极限点},后者为公切线
    上的实际入啮点与脱啮点\hint{实际上是与齿顶圆的交点}。
    \item 连续传动条件\hint{始终有齿轮啮合}:实际啮合线长度大于基圆齿距 $p_b$,二者之比
    为重合度 $\varepsilon\geq1$;一般机械要求重合度在 $1.1$ 至 $1.4$。
    \item 图解重合度:测出极限啮合线长度与实际啮合线长度,求比值。
    \item \emph{重合度与其他参数关系:与齿数 $z_1$、$z_2$ 正相关\hint{「多子多福」},
    与模数 $m$ 无关。\hint{公式不用记,但需掌握作图法计算 $\varepsilon$。}}
    \item 轮齿的相对滑动:不在节点啮合时,两齿廓上速度相异导致相对滑动。
    \begin{itemize}\tightlist
        \item 啮合点运动方向:主动轮上沿节线向外,从动轮上沿节线向内。
        \item 越靠近齿根、齿顶处,相对滑动越剧烈\hint{运动方向差异最大}。
        \item 为避免相对滑动,应使实际啮合线远离极限啮合点\hint{与提高重合度的目标有
        冲突}。
    \end{itemize}
    \item 标准中心距:$a=r_1+r_2$,即节圆与分度圆重合时中心距;标准安装:一对标准齿轮
    按标准中心距安装。
\end{itemize}

\subsection{齿轮传动的制造}
\begin{itemize}\tightlist
    \item 齿轮的结构:轮缘、轮辐、轮毂。制造方式:锻造\hint{直径较小时适用,常与轴一同
    制作为一体的「齿轮轴」}、铸造\hint{直径 $d_a>\SI{500}{mm}$时采用}、镶圈\hint{对大
    直径齿轮采用,可节约贵重金属}、焊接\hint{对单个或小批量,或尺寸太大的齿轮采用}。
    \item 加工方式:\minor{铸造法、热轧法、冲压法、}切削法\hint{仿型法、展成法}。
    \begin{itemize}\tightlist
        \item 仿型法:用各类铣刀一次性铣出齿廓形状。\hint{不精准,且对不同参数齿轮都要
        重新设计刀具。}
        \item 展成法:相当于将一堆共轭齿轮之一变为刀具\hint{插刀、滚刀},在配合时不断
        包络出轮坯上的齿廓。\hint{根据齿轮正确啮合条件,同一刀具适用于同一模数的各类齿
        轮,通用性强。}
        \item 相较于插刀,滚刀可实现连续切削\hint{不用退刀},效率更高。
    \end{itemize}
    \item 根切:刀具齿顶线超过极限啮合点,导致齿廓的下半部分被切掉,削弱轮齿强度并导致
    重合度降低。避免根切的条件:$z\geq\frac{2h_a^\star}{\sin^2\alpha}=z_\text{min}$,
    对 $h_a^\star=1$、$\alpha=\SI{20}\degree$ 可算得 $z_\text{min}=17$。
    \item 变位齿轮:为特定目的,改变齿条与轮坯间距,导致切出的不是标准齿轮。
    \begin{itemize}\tightlist
        \item 间距改变量按模数换算为 $mx$,$x$ 称为变位系数。
        \item 正变位:刀具远离轮坯,$x>0$,可在 $z<z_\text{min}$ 的情况下避免根切。
        \item 负变位:刀具靠近轮坯,$x<0$,可用于调整大齿轮的中心距。
    \end{itemize}
    \item 齿轮常见失效形式:
    \begin{enumerate}\tightlist
        \item 轮齿折断:分疲劳折断与过载折断,主因为强度不足、应力集中。\hint{增大齿根
        强度是关键,可增大圆角半径、喷丸处理等。}
        \item 齿面点蚀:轮齿表面的接触疲劳破坏,常出现在节线附近\hint{下侧靠近齿根位置}。闭式齿轮传动中居多。
        润滑油的渗入会推波助澜。\hint{可提高齿面硬度、降低齿面粗糙度、采用高黏度润滑油等。}
        \item 齿面胶合:齿面在高温、高压下直接黏着,导致材料被撕落。常见于重载低速场合。
        \hint{可对表面热处理、采用高黏度润滑油或掺入抗胶合剂、降低齿面粗糙度、合理搭配
        齿面材料等。}
        \item 齿面磨损:由磨粒或两轮齿硬度差造成的磨损,最终一般导致轮齿折断;开式齿轮
        传动中居多。\hint{改为闭式传动是最有效的方法,也可提高齿面硬度、保持润滑油清洁等。}
        \item 齿面塑形流动:由齿面相对滑动所致,导致齿面材料发生迁移;表现为主动轮节线
        处凸起,从动轮则凹陷\hint{可参考相对滑动方向}。\hint{可提高齿面硬度、采用高黏度
        润滑油等。}
    \end{enumerate}
    \item 常用齿轮材料:锻钢\hint{应用最广,一般需热处理}、铸铜\hint{用于不易锻造的大型
    毛坯}、铸铁\hint{用于铸成形状复杂的毛坯,性能较差,只适用于低速、平稳载荷}、非金属材
    料\hint{常用于高速、小功率、常温场合}。
\end{itemize}

\subsection{齿轮强度分析}
\begin{itemize}\tightlist
    \item 齿轮传动的强度设计与强度计算路线:
    \begin{itemize}\tightlist
        \item 强度计算:根据实际情况折合强度条件\then 分析载荷\then 根据实际情况折合
        计算载荷\then 计算各项应力\then 校核。
        \item \minor{强度设计:选取材料与热处理方式\then 折合强度条件\then 初选载荷系
        数,获得计算载荷\then 初选若干参数,按强度条件求确定其他参数\then 强度计算
        \hint{见上}。}
    \end{itemize}
    \item 名义载荷 $F_n$:理想状态下的载荷;计算载荷 $F_{ca}$:按工作情况进行修正。
    \[F_{ca}=KF_n=K_AK_vK_\alpha K_\beta F_n\]
    式中,$K>1$ 为总的载荷系数,其他各项\hint{均大于 1} 为:
    \begin{itemize}\tightlist
        \item 使用系数 $K_A$:与原动机的过载、振动、冲击有关,冲击越严重则该系数越高。
        \item 动载系数 $K_v$:因啮合不精准而导致内部附加动载荷,与轮齿精度\hint{分 12
        级,常用 6--8 级}、齿面硬度等有关。
        \item 齿面载荷分布系数 $K_\beta$:因载荷在空间上分布不均所致,对轴上齿轮副必须
        考虑,与齿轮在轴上位置、轴的刚度等有关。
        \item 齿间载荷分布系数 $K_\alpha$:因多对轮齿的载荷分布不均所致,与精度等级、
        重合度 $\varepsilon$ 等有关。
    \end{itemize}
    以上所有系数均可查图表确定。
    \item 轮齿受力分析:常将载荷简化为「作用在齿宽中点的集中力」,分解为圆周力 $F_t=
    2000\frac{T_1}{d}$ \hint{$T_1=9550\frac{P}{n_1}$ 为齿轮转矩,$d_1$ 为节圆直径}
    与径向力 $F_r=F_t\tan\alpha$。二者合成即为名义载荷 $F_n$。
    \item 计算准则:主要针对折断\hint{弯曲疲劳强度}、点蚀\hint{接触疲劳强度}。
    \begin{itemize}\tightlist
        \item \emph{考这些公式的计算,算我输。}
        \item 弯曲疲劳强度:假设全部载荷作用于齿顶\hint{保守假设},用「\SI{30}\degree
        切线法」
        \footnote{即在齿根处,作两条与轮齿对称轴夹 \SI{30}\degree 的直线与齿根处齿廓
        内切,两切点之连线即标示出危险截面位置。}
        确定危险截面处齿厚 $s_F$,进而按悬臂梁模型校核:
        \[ \sigma_F=\frac{2000KT_1}{d_1b}\cdot Y_{Fa}Y_{Sa}Y_\varepsilon\leq
        [\sigma_F] \]
        其中齿形系数 $Y_{Fa}$、应力修正系数 $Y_{Sa}$ 查图确定,重合度系数
        $Y_\varepsilon=0.25+\frac{0.75}\varepsilon$ 直接计算。
        \item 接触疲劳强度:整合相关参数,代入赫兹公式可求得
        \[ \sigma_H=\frac{Z_EZ_HZ_\varepsilon}{a}\sqrt{\frac{500KT_1(u+1)^3}{bu}}
        \leq[\sigma_H] \]
        其中弹性系数 $Z_E$ 与节点区域系数 $Z_H$ 查表求取,$Z_\varepsilon=\sqrt{\frac{
        4-\varepsilon}3}$ 可直接计算。
        \item \emph{影响因素分析:$\sigma_F$ 与齿面面积 $bm$ 反相关,$\sigma_H$ 与
        $a,b$ 反相关但与 $m$ 无关。}
    \end{itemize}
    \item 强度条件的折合:
    \begin{itemize}\tightlist
        \item 弯曲疲劳强度:$[\sigma_F]=\frac{Y_{ST}Y_N}{S_{F\lim}}\sigma_{F\lim}$,
        其中 $\sigma_{F\lim}$ 为疲劳强度极限,$Y_{ST}=2$ 为试验齿轮的修正系数,$Y_N$
        为寿命系数。
        \item 接触疲劳强度:$[\sigma_H]=\frac{Z_N}{S_{H\lim}}\sigma_{H\lim}$,
        其中 $\sigma_{F\lim}$ 为疲劳强度极限,$Y_N$为寿命系数。
    \end{itemize}
    以上系数均可查图表确定。
    \item 强度设计:分别校核两个齿轮的弯曲强度 $\sigma_{F1}$、$\sigma_{F2}$ 和它们的
    接触强度 $\sigma_H$。
    \begin{itemize}\tightlist
        \item 对软齿面闭式传动,按 $[\sigma_H]$ 设计、按 $[\sigma_F]$ 校核.
        \item 对硬齿面闭式传动,按 $[\sigma_F]$ 设计、按 $[\sigma_H]$ 校核。
        \item 对开式传动,只考虑 $[\sigma_F]$,且应将 $[\sigma_F]$ 降低 25\%--30\%
        \hint{计入磨损}。 
    \end{itemize}
    \item 齿宽系数:$\phi_d=b/d_1$、$\phi_a=b/a$。
    \item 参数设计:
    \begin{itemize}\tightlist
        \item 来自于强度校核:
        \begin{gather}
        m\geq\sqrt{\frac{2000KT_1Y_\varepsilon}{\phi_dz_1^2}\left(\frac{Y_{Fa}
        \cdot Y_{Sa}}{[\sigma_F]}\right)}\\
        a\geq(u+1)\sqrt[3]{\frac{500KT_1}{\phi_au}\left(\frac{Z_EZ_HZ_\varepsilon
        }{[\sigma_H]}\right)^2}
        \end{gather}
        \item 齿数 $z$:软齿轮取 $z_1$ 为 20--40,硬齿轮取 $z_1$ 为 12--17。
        \item 齿宽系数:由于 $\phi_a/\phi_d=\frac{2}{u+1}$,两个系数中只需确定一个,
        可查表。
        \item 齿数比 $u$:不宜过大,否则应采用蜗轮或轮系传动。
    \end{itemize}
\end{itemize}

\section{蜗杆传动}
\begin{itemize}\tightlist
    \item 蜗杆传动由齿轮传动演化而来。
    \item 蜗杆传动的组成:蜗杆(1)与蜗轮(2),一般蜗杆主动\hint{减速传动}。
    \item 优点:传动比高;重合度大,传动平稳;具有自锁效应,可防止逆行。
    \item 缺点:传动效率低,发热严重;成本高\hint{蜗轮常用青铜制造,是贵重金属}。
    \item 一般置于系统的高速级,用于第一轮减速、传递中小功率。可用作分度机构
    \hint{高传动比}。
\end{itemize}

\subsection{蜗杆传动基本概念}
\begin{itemize}\tightlist
    \item 蜗杆传动的分类\hint{按蜗杆外廓}:圆柱蜗杆\hint{最常见}\minor{、环面蜗杆、
    锥蜗杆\hint{后两者重合度高,传动更平稳}}。
    \item 蜗杆与蜗轮分左旋与右旋,右旋居多;只有同旋向才能配合。
    \item 普通圆柱蜗杆分类:阿基米德蜗杆\hint{易于制造;磨削困难,精度较低}、渐开线蜗杆
    \hint{可以磨削,用于高精度传动}、法向直廓蜗杆\hint{不易磨削}、锥面包络蜗杆\hint{用
    展成法铣制而成,便于磨削}。
    \item 空间方位标记:轴向为 $a$,法向为 $n$,端面或周向为 $t$。
    \item 蜗杆基本几何参数:螺旋角 $\beta_1,\beta_2$;导程角 $\gamma=\SI{90}\degree
    -\beta_1$;模数 $m$ 与压力角 $\alpha$
    \footnote{蜗轮、蜗杆应满足正确啮合条件,即应具有相同的模数与压力角,具体到中间平面上,
    是蜗杆的轴面模数 $m_{a1}$、轴向压力角 $\alpha_{a1}$ 与蜗轮的端面模数 $m_{t2}$、
    端面压力角 $\alpha_{t2}$ 对应相等。};蜗杆头数 $z_1$\hint{螺旋线条数}与蜗轮齿数
    $z_2$;蜗杆分度圆直径 $d_1$ 与蜗轮分度圆直径 $d_2$;中心距 $a$。
    \begin{itemize}\tightlist
        \item 对于阿基米德蜗杆,法向压力角 $\alpha_n=\SI{20}\degree$ 已标准化,轴向压
        力角按 $\tan\alpha_a=\frac{\tan\alpha_n}{\cos\gamma}$ 计算。
        \item 因效率、制造难度等因素,蜗杆头数 $z_1$ 一般取 1、2、4、6。
        \item $z_2$ 一般取 32 -- 80,不宜过高\hint{分度传动除外}。
        \item 轴向齿距 $p_a$\hint{不分螺旋线} 与导程 $p_z$\hint{同一螺旋线}:$p_z=z_1p_a$。
        \item 螺旋线关系:$d_1=m\frac{z_1}{\tan\gamma}$。
        \item 蜗杆直径系数 $q=\frac{d_1}{m}$:将蜗杆直径按模数标准化,不一定为整数。
        \item 中心距可按 $a=\frac12(d_1+d_2)=\frac12(q+z_2)m$ 折算为模数。
    \end{itemize}
    \item 蜗杆转动传动比:$i=\frac{\omega_1}{\omega_2}=\frac{z_2}{z_1}$\hint{蜗杆
    主动时为减速传动}。
    \item 转向判定\hint{对蜗轮蜗杆均适用}:判断旋向\hint{顺着轴看,谁高向谁旋}\then
    采用对应的「左/右手准则」\then 大拇指所指方向的反向为速度方向\then 进而确定转向。
\end{itemize}

\subsection{蜗杆传动的受力与传动分析}
\begin{itemize}\tightlist
    \item 齿面间的相对滑动:$v_s=\sqrt{v_1^2+v_2^2}$,其中 $v_1$ 与 $v_2$ 分别为蜗杆的
    径向与周向速度\hint{$\frac{v_2}{v_1}=\tan\gamma$};可导致磨损与胶合。
    \item \emph{由于 $v_s$ 的存在,传动比 $i\neq\frac{d_2}{d_1}$。\hint{实际为 $i=\frac{d_2}q$}。}
    \item 受力分析:载荷 $F_n$ 可分为两个轴向力、两个法向力、两个周向力,对应相等:
    \begin{gather}
    F_{t1}=F_{a2}\minor{=F_n\cos\alpha_n\sin\gamma+fF_n\cos\gamma}
    =\frac{2000T_1}{d_1}\quad\text{(互为反作用力)}\\
    F_{a1}=F_{t2}\minor{=F_n\cos\alpha_n\cos\gamma-fF_n\sin\gamma}
    =\frac{2000T_2}{d_2}\quad\text{(互为反作用力)}\\
    F_{r1}=F_{r2}=F_n\sin\alpha_n\approx F_{a1}\tan\alpha
    \end{gather}
    其中 $T_1$ 与 $T_2$ 为两构件上的转矩,满足 $T_2=T_1i\eta_1$\hint{$\eta_1$ 为啮合效
    率}。\minor{只有已知 $T_1$ 与 $T_2$ 后才能确定所有的力。}
    \item 受力方向分析:
    \begin{itemize}\tightlist
        \item 径向力 $F_r$:均指向各自的轴心。
        \item 周向力 $F_t$:主动件周向力阻碍转动\hint{受从动件阻力},从动件周向力推动
        转动\hint{被主动件推动}。
        \item 轴向力 $F_a$:按对应反作用力\hint{对方周向力}方向取反;\minor{对于主动
        件,其轴向力也可按「左/右手定则」确定\hint{从动件的则与「定则」结果相反}}。
    \end{itemize}
    \item 啮合效率 $\eta_1$ 与摩擦角:按功率分析有
    \begin{equation}
    \eta_1=\frac{F_{t2}v_2}{F_{t1}v_1}=\frac{\tan\gamma}{\tan(\gamma+\phi_v)}
    \end{equation}
    其中 $\phi_v=\tan\left(f/\cos\alpha\right)=\tan f_v$ 称为当量摩擦角,$f_v$
    称为当量摩擦因数;均可按材料和 $v_s$ 的值查表获得。
    \item 效率 $\eta_1$ 与 $\gamma$、$f_v$ 的关系:
    \begin{itemize}\tightlist
        \item $\eta_1$ 关于 $\gamma$ 先升后降,一般取 $\gamma<\SI{30}\degree$;
        \item $\eta_1$ 与 $f_v$ 反相关,故应尽量降低当量摩擦因数。
    \end{itemize}
    \item \emph{$v_s$ 增大时,润滑油可充分扩散,进而使 $f_v$ 与 $\phi_v$下降,因此蜗
    杆常用于高速级以提高效率。布置在高速级还使同功率下受力降低,进而减小尺寸、节省材料。}
    \item 自锁现象:蜗轮主动时,两构件周向力之比可算得为 $F_{t1}/F_{t2}=\tan(\gamma-
    \phi_v)$,当 $\gamma\leq\phi_v$ 时意味着主动件不能驱动从动件,达成自锁\hint{此时
    $\eta_1<0.5$,不宜传动}。
    \item 蜗杆传动总效率:$\eta=\eta_1\eta_2\eta_3$
    \begin{itemize}\tightlist
        \item $\eta_2$ 与轴承中摩擦的损耗相关,对每对滚动轴承取 0.99 -- 0.995,对每对
        滑动轴承取 0.97 -- 0.99。
        \item $\eta_3$ 与搅油\hint{润滑油}损耗相关,一般取 0.98.
        \item 一般常用 $\eta=(100-3.5\sqrt{i})\%$ 事先估计,以便分析和设计。
    \end{itemize}
\end{itemize}

\subsection{蜗杆传动设计}
\begin{itemize}\tightlist
    \item 蜗杆传动的失效形式:与齿轮传动类似,一般蜗轮轮齿首先失效;蜗轮齿面的胶合、磨
    损、点蚀居多。
    \item 设计准则:与齿轮类似,考虑齿面接触强度 $[\sigma_H]$ \hint{点蚀}与齿根弯曲
    强度 $[\sigma_F]$ \hint{疲劳折断},以及热平衡计算\hint{胶合}。
    \begin{itemize}\tightlist
        \item 闭式传动:按 $[\sigma_H]$ 设计,按 $[\sigma_F]$ 校核,要做热平衡计算。
        \item 开式传动:只考虑 $[\sigma_F]$。
    \end{itemize}
    \item 材料搭配:一硬一软\hint{减摩性好},\emph{一般用钢制蜗杆与青铜蜗轮}。
    \begin{itemize}\tightlist
        \item \minor{蜗杆材料:优质碳素钢、合金钢等。}
        \item \minor{蜗轮材料:铸锡青铜\hint{性能最好,易加工,价格贵}、铸铝铁青铜
        \hint{性能一般,强度高,价格低}、灰铸铁\hint{成本低,可用在低速或大尺寸场合}。}
    \end{itemize}
    \item 减摩性与耐磨性的区别:前者针对一对材料,后者针对一种材料。
    \item 强度设计与校核:设计时的 $m^2d_1$ 为辅助量,在查表选取标准值时可用。
    \begin{itemize}\tightlist
        \item 接触强度:校核公式为
        \begin{equation}
        \sigma_H=15900\sqrt{\frac{KT_2}{m^2d_1z_2^2}}\leq[\sigma_H]
        \end{equation}
        \minor{其中的 15900 针对铸锡青铜,对铸铝铁青铜与灰铸铁应分别换成 16400 与
        16600。}$K$ 为载荷系数,与蜗轮速度正相关;平稳时取 1 -- 1.25,波动时取 1.2
        -- 1.4。设计公式为
        \begin{equation}
        m^2d_1\geq\left(\frac{15900}{z_2[\sigma_H]}\right)^2KT_2.
        \end{equation}
        \item 弯曲强度:校核公式为
        \begin{equation}
        \sigma_F=\frac{2000KT_2}{d_1d_2m\cos\gamma}\leq[\sigma_F],
        \end{equation}
        设计公式为
        \begin{equation}
        m^2d_1\geq\frac{2000KT_2Y_{Fa2}}{z_2[\sigma_F]\cos\gamma}.
        \end{equation}
        其中的齿形系数 $Y_{Fa2}$ 可按当量齿数 $z_{v2}=\frac{z_2}{\cos^3\gamma}$
        查表。
        \item 热平衡计算:验算温升
        \begin{equation}
        \Delta t=\frac{1000P(1-\eta)}{K_tA}\leq[\Delta t]
        \end{equation}
        其中 $P$ 为功率,$A$ 为散热面积\hint{一般估计为散热片表面积的一半},$K_t$ 为
        散热系数\hint{ 一般取 10 -- \SI{17}{J/(n^2.s.\celsius)}};许用温升 $[\Delta
        t]$ 为 30 -- \SI{40}\celsius,且油温应小于 \SI{80}\celsius。
    \end{itemize}
    \item 强度条件确定:
    \begin{itemize}\tightlist
        \item 许用接触应力 $[\sigma_H]$:对低强度青铜蜗轮,根据接触应力循环次数 $N_H$
        确定:$[\sigma_H]=[\sigma_{H0}]\cdot\sqrt[8]{10^7/N_H}$,$[\sigma_{H0}]$ 为
        基本许用接触应力。\minor{$N_H$ 的下限取 $2.6\times10^5$,上限取 $25\times
        10^7$。}对高强度青铜,直接根据 $v_s$ 查表。
        \item 许用弯曲应力:根据弯曲应力循环次数 $N_F$ 确定:$[\sigma_F]=[\sigma_{F0}]
        \cdot\sqrt[9]{\frac{10^6}{N_F}}$,$[\sigma_{F0}]$ 为基本许用弯曲应力。\minor{
        $N_F$ 的下限取 $10^5$,上限取 $25\times10^7$。}
    \end{itemize}
    \item 蜗杆结构:直径较小时,做成一体的蜗杆轴;直径较大\hint{$d_{f1}/d\geq1.7$}
    时分别制造。
    \item \minor{蜗轮结构:齿圈式、螺栓连接式、整体式、镶铸式。}
    \item \minor{蜗杆的冷却手段:增设散热片\hint{沿气流方向设置};在蜗杆轴上安装风扇;
    在箱体油池内安装蛇形冷却水管等。}
\end{itemize}

\section{轮系}
\begin{itemize}\tightlist
    \item 轮系的分类:定轴轮系\hint{所有齿轮轴都固定于机架}、周转轮系\hint{至少一个齿
    轮轴不固定}、混合轮系\hint{前二者的组合}。
    \item 平面定轴轮系\hint{轴线相互平行}与空间定轴轮系\hint{一般含锥齿轮、蜗杆传动}。
    \item \emph{周转轮系与混合轮系在实质上不能算作两类,只是层次上的划分。}
    \item 常见概念:双联齿轮\hint{两个齿轮同轴固结而成的构件,转速相同}、惰轮\hint{同时与两个齿
    轮啮合,只改变速度方向}。
    \item 轮系分析的主要目的:
    \begin{itemize}\tightlist
        \item 弄清轮系中各部分的传动比\hint{大小、方向};
        \item 分析轮系的运动规律\hint{由对传动比的分析实现}。
    \end{itemize}
\end{itemize}

\subsection{定轴轮系}
\begin{itemize}\tightlist
    \item 箭头标注规则:
    \begin{enumerate}\tightlist
        \item 平行轴外啮合齿轮:箭头方向相反。
        \item 平行轴内啮合齿轮:箭头方向相同。
        \item 锥齿轮:箭头同时朝内或朝外。
        \item 蜗杆传动:按蜗杆传动的方法分析。
    \end{enumerate}
    \item 对平面定轴轮系,以及输入、输出轴平行的轮系,可用传动比中正负号表示输入与输出
    速度的方向是否相同。
    \item 传动比大小的计算通式:
    \begin{equation}
    |i_\text{主,从}|=\left|\frac{n_\text{主}}{n_\text{从}}\right|=\frac{\text{各
    从动轮齿数连乘积}}{\text{各主动轮齿数连乘积}}
    \end{equation}
\end{itemize}

\subsection{周转轮系与混合轮系}
\begin{itemize}\tightlist
    \item 概念:太阳轮/中心轮\hint{轴线固定}、行星轮\hint{轴线不固定}、行星架/系杆
    \hint{连接太阳轮与行星轮之轴}。
    \item \emph{每个周转轮系中有且仅有一个行星架,可通过行星架的数量确定周转轮系数量。}
    \item 常见周转轮系:行星轮系\hint{$F=1$}、差动轮系\hint{$F=2$}。
    \item 周转轮系的速度分析:「反转法」\hint{与凸轮机构有所相似}
    \begin{itemize}\tightlist
        \item 给系统附加反向的 $-n_H$,此时行星架固定下来,行星轮仅有自转,转化为一个
        定轴轮系\hint{原有轮系的转化轮系};
        \item 传动比应改写为
        \begin{equation}
        i'_{12}=\frac{n_1^H}{n_2^H}=\frac{n_1-n_H}{n_2-n_H}=-\frac{z_2}{z_1}
        \end{equation}
        在若干参数已知时可由此方程解出其他参数。
        \item 转化轮系中的速度箭头应用虚线表示,实际方向应根据计算结果确定
        \footnote{类似于力学与电路分析中的参考/假想方向、实际方向之关系}。
    \end{itemize}
    \item 混合轮系分析流程:分解轮系为若干基本轮系\hint{通过行星轮对应的行星架确定}\then
    分别分析,列出方程\then 联立求解。
    \item 轮系的功用:
    \begin{enumerate}\tightlist
        \item 实现较远距离的传动。
        \item 逐级叠加,获得较大传动比。
        \item 通过轮系的组合,实现变速与换向。
        \item \minor{运动分解与合成\hint{实现运算}。}
        \item 实现分路传动\hint{用一个原动机实现多重运动}。
        \item \minor{实现特殊的运动轨迹\hint{旋轮线,包括椭圆、内摆线、纽扣线等}。}
    \end{enumerate}
    \item \minor{「新型」轮系:RV 减速器、谐波齿轮传动等。}
\end{itemize}


\section{带传动}
\begin{itemize}\tightlist
    \item 带传动的三个组成部分;刚性传动与挠性传动;两类带传动。
    \item 三类主要传动形式\hint{及适用的速度范围};四种带\hint{形状、实际工作面、优缺点}。
    \item 当量摩擦因数\hint{谁的?计算公式、含义}。
    \item 五个主要几何参数、它们的关系\hint{近似计算公式}。
    \item 带传动的六个特点\hint{减震、打滑、效率、结构与寿命、尺寸、传动比};应用范围
    \hint{功率、带速}。
\end{itemize}

\subsection{V带与带轮}
\begin{itemize}\tightlist
    \item 三种V带;V带的组成\hint{四个部分}。
    \item 普通V带的参数\hint{型号——四项几何尺寸};轮槽楔角\hint{与V带楔角的关系}。
    \item 带的节面\hint{定义}与带轮基准直径。
    \item \minor{另两种V带的应用场合\hint{功率、尺寸、工况}。}
    \item \minor{带轮的设计要求\hint{质量、安装、平衡性、表面};}带轮材料
    \hint{与速度关系};\minor{四种结构形式\hint{及选用依据}。}
\end{itemize}

\subsection{工作原理}
\begin{itemize}\tightlist
    \item 基本工作原理\hint{带轮如何驱动};二力相抗原理\hint{哪两个力?随载荷加大的变
    化趋势};带传动设计的主要问题。
    \item 张紧力、初拉力$F_0$;紧边与松边\hint{识别、形成原理};紧边拉力$F_1$、松边
    拉力$F_2$\hint{方向、大小关系};不变形条件\footnote{经常被遗忘。}
    \hint{$F_0$、$F_1$、$F_2$的等式};有效工作拉力$F_e$、摩擦力$F'$\hint{关系、平衡
    原理、计算公式};打滑原理\hint{谁顶不住了?}。
    \item \emph{一般总是小带轮发生打滑,其包角应尽可能的大。}
    \item 欧拉公式\hint{$F_1=F_2\mathrm{e}^{f\alpha}$,说了什么?};最大有效工作
    拉力$F_{ec}$\hint{计算公式,从$F_1$到$F_0$};打滑位点\hint{大或小?}。
    \item 避免打滑的三点原则\hint{来自于$F_{ec}$的计算公式};\minor{为什么铸铁优于
    钢。}
    \item 三种应力:拉应力\hint{估算方法};弯曲应力\hint{估算方法、谁大?};离心拉应力
    \hint{均匀、估算方法};应力分布规律\hint{绘出示意图};应力最大点;带的疲劳强度条件。
    \item 弹性滑动\hint{形成原理、方向};速度不等式$v_1>v_\text{带}>v_2$\hint{为什
    么?};滑动率\hint{计算公式、大致数值};传动比不恒定\hint{计算公式};弹性滑动的
    影响\hint{效率、发热、速度}。
    \item 滑动弧、静弧;打滑与滑动弧的关系;打滑的危害\hint{磨损、发热、速度、传动}。
\end{itemize}

\subsection{校核与设计}
\begin{itemize}\tightlist
    \item 四种失效形式\hint{打滑、疲劳、磨损、\minor{带轮断裂}};主要设计依据:
    「一个保证,两个足够」\hint{保证不打滑,疲劳强度及寿命足够}。
    \item 强度计算方法\hint{基本条件、$[\sigma]$条件\then 关于$F_e$的条件\then 功率
    条件};最佳带速\hint{功率随速度的变化规律};基本额定功率$P_0$\hint{条件、表格构成}。
    \item 一般条件下的功率校核:额定功率$[P]=(P_0+\Delta P_0)K_\alpha K_L$,其中
    $\Delta P_0=K_bn_1(1-1/K_i)$\hint{各项含义、随相关参数的变化规律、表格构成}。
    \item 设计计算步骤:
    \begin{enumerate}\tightlist
        \item 条件\hint{工况、带速或传动比、功率};
        \item 计算功率$P_d=K_AP$\hint{含义};
        \item 初选型号\hint{查图};
        \item 确定带轮基准直径\hint{小带轮查表过限,大带轮按传动比圆整};
        \item 验算带速\hint{多少为宜};
        \item 确定$a$及$L_d$\hint{不等式初选$a$,估算$L_d$并查表圆整,再近似反算$a$};
        \item 验算小带轮包角\hint{公式;一般要求};
        \item 确定根数$z$\hint{计算载荷、额定功率};
        \item 计算初拉力$F_0$\hint{公式};
        \item 计算压轴力$F_P$\hint{受力分析,导出公式}。
    \end{enumerate}
    \item \minor{张紧装置设计\hint{滑道、摆架、重力自动张紧、张紧轮}。}
    \item 使用与维护原则\hint{同心平行对中;带在轮槽中的位置;成组时尺寸均匀……}。
\end{itemize}

\section{间歇传动机构}
\subsection{棘轮机构}
\begin{itemize}\tightlist
    \item 组成部分\hint{摇杆、棘爪、棘轮、止回爪、机架};运动原理\hint{摆动「拖刀」};
    转换方向\hint{摆动\then 间歇转动}。
    \item 分类:齿式/摩擦式、单向/可换向、外啮合/内啮合、齿式/双动式。
    \item 特点:
    \begin{itemize}\tightlist
        \item 齿式:工作可靠,精度高;有硬冲,噪声、磨损严重。
        \item 摩擦式:工作平稳,无级调整;精度差。
    \end{itemize}
    \item 应用场合:转位分度与送进\hint{牛头刨刀}、止动\hint{起重}、超越离合器\hint{
    自行车后轴——原理}。
    \item 齿式棘轮机构设计:模数$m=\frac{d_a}{z}$\hint{基准在齿顶圆上},齿数$z$
    \hint{按分度确定,一般取$8\sim60$},轮齿倾斜角$\alpha$\hint{条件$\alpha>\varphi$
    ,一般取$\alpha=\SI{20}{\degree}$}。
\end{itemize}

\subsection{槽轮机构}
\begin{itemize}\tightlist
    \item 组成部分\hint{拨盘、槽轮};运动原理;转换方向\hint{连续转动\then 间歇不等速
    转动}。
    \item 分类:外槽轮/内槽轮\hint{记住形状}。
    \item 特点:工作可靠且较平稳,效率高;有柔冲,分度难调整\hint{受工艺限制,槽数不宜
    过多}。
    \item 运动分析与设计准则:
    \begin{itemize}\tightlist
        \item 运动系数:$k=\frac{\text{运动时间}}{\text{总时间}}$,外槽轮$k=\frac12
        -\frac1z<0.5$,内槽轮$k=\frac12+\frac1z>0.5$。
        \item 意义:若要使外槽轮$k>0.5$,应采用$n>1$个拨销使$k=n\left(\frac12-\frac%
        1z\right)>0.5$\hint{但不能超过$1$,由此可解出$n$的最大值};增大$z$也可适当增加
        运动系数;单拨销外槽轮的$z>3$才能正常运转\hint{否则$k\leq0$}。
        \item 齿数设计:一般取$z=4,6,8$。
        \item \minor{运动特性:槽轮转动非等速;角速度与角加速度取决于$z$。}
    \end{itemize}
\end{itemize}

\subsection{不完全齿轮机构}
\begin{itemize}\tightlist
    \item 组成部分\hint{两轮;凹/凸锁止弧};运动原理\hint{有齿时啮合传动,无齿时锁止};
    转换方向\hint{连续转动\then 间歇等速转动}。
    \item 分类:外啮合、内啮合。
    \item 特点:结构简单、分度设计灵活、精度高;有硬冲\hint{继承于齿轮,可用附加杆装置
    减缓冲击}。
\end{itemize}


\section{机械系统动力学}
\begin{itemize}\tightlist
    \item 三大问题:机械真实运动;飞轮调节波动;转子平衡问题。
    \item \minor{原理部分:设计的灵魂。}
    \item \minor{现代设计方法:以动力学设计囊括静态设计。}
\end{itemize}

\subsection{机械真实运动分析}
\begin{itemize}\tightlist
    \item 作用在机械上的力:\minor{重力、惯性力、约束反力、摩擦力、}驱动力、工作阻力
    \footnote{摩擦力并不被视为工作阻力,而被视为「有害阻力」。}
    \item 力的机械特性:力--运动参数\hint{机械特性曲线}
    \item 常见机械特性:
    \begin{itemize}\tightlist
        \item 驱动力:常数\hint{液压活塞}、位置函数\hint{内燃机}、速度函数\hint{电动机}
        \item 工作阻力:常数\hint{起重机}、位置函数\hint{压缩机}、速度函数\hint%
        {鼓风机、离心泵}、时间函数\hint{破碎机}。
        \item \minor{特性的确定:其他课程中解决。}
    \end{itemize}
    \item 机械运动的阶段\hint{三个}:启动、稳定运动\hint{主动件保持常速或周期波动——
    匀速稳定运动、变速稳定运动}、停车;正常工作速度
    \item 功能分析:$W_a-W_c=\frac12\Delta\sum(m_iv_{c_i}^2+J_{c_i}\omega^2)$;
    分阶段情况\hint{$W_a-W_c>0$\then$=0$\then$<0$}。
    \item 等效动力学模型:等效构件、等效力\hint{或力矩}、等效质量\hint{或转动惯量};
    分析方法:功率相等原则、动能相等原则;其他事项\hint{常将$F_a$与
    $F_c$分开分析}。
    \begin{gather}
    \left\{
    \begin{aligned}
    F_\text{v}&=\sum_iF_i\frac{v_i\cos\alpha_i}{v}+\sum_i\left(\pm M_i\frac%
    {\omega_i}{v}\right)\\
    m_\text{v}&=\sum_i\left[m_i\left(\frac{v_{c_i}}{v}\right)^2+J_{c_i}\left(
    \frac{\omega_i}{v}\right)^2\right]
    \end{aligned}\right.\\\left\{
    \begin{aligned}
    M_\text{v}&=\sum_iF_i\frac{v_i\cos\alpha_i}{\omega}+\sum_i\left(\pm M_i
    \frac{\omega_i}{\omega}\right)\\
    J_\text{v}&=\sum_i\left[m_i\left(\frac{v_{c_i}}{\omega}\right)^2+J_{c_i}\left(
    \frac{\omega_i}{\omega}\right)^2\right]
    \end{aligned}\right.
    \end{gather}
    \item 等效分析的结果\hint{速度比的函数,仅与位置有关;不等于简单的求和\minor{,除非
    ……}}。
    \item 等效分析的意义:使系统仅剩一个自由度,便于求解。
    \begin{equation}
    F_\text{v}=m_v\frac{\mathrm{d}v}{\mathrm{d}t}
    \end{equation}
\end{itemize}

\subsection{速度波动与飞轮设计}
\begin{itemize}\tightlist
    \item 调速原理:增大构件的质量或转动惯量\then 安装飞轮\hint{起水库之用};飞轮的
    影响\hint{降低速度波动幅值、减少原动机输出功率;延长启动与制动时间}。
    \item \emph{一般将飞轮装在高速轴上,可减轻飞轮的质量及尺寸。}
    \item 平均角速度$\omega_m$:实际平均角速度\hint{积分平均}与算术平均角速度;不均匀
    系数$\delta=\frac{\omega_\text{max}-\omega_\text{min}}{\omega_\text{m}}$及
    许用不均匀系数$[\delta]$\hint{大致量级;利用$[\delta]$反求速度波动许可值}
    \item 飞轮设计原理:计算$J_F\approx J_\text{v}$\hint{飞轮比机械自身转动惯量大
    很多},保证由
    \begin{equation}
    W_y=\int_{\varphi_{E_\text{min}}}^{\varphi_{E_\text{max}}}M_\text{v}\,
    \mathrm{d}\varphi=\frac12J_F(\omega_\text{max}^2-\omega_\text{min}^2)
    =J_F\omega_m^2\delta
    \end{equation}
    所确定的$\delta\leq[\delta]$,故应有
    \begin{equation}
    J_F=\frac{W_y}{\omega_\text{m}^2\delta}.
    \end{equation}
    \item 最大盈亏功$W_y$的确定:$M_\text{v}-\varphi$图
    \footnote{一般为$M_\text{va}$与$M_\text{vc}$,此时应求两曲线所夹面积(即作差)。}
    \then 能量指示图\hint{曲线包围面积}\then 最大盈亏功\hint{最值点之差}
    \item \minor{非周期性速度波动:调速器}
\end{itemize}

\subsection{转子的动平衡与静平衡}
\begin{itemize}\tightlist
    \item 平衡问题的意义:减小振动\hint{包括共振}造成的危害;三类平衡问题\minor{
    \hint{刚性转子、挠性转子、机械在基座上的平衡}}。
    \item 临界转速$n_c$\hint{第一阶,第二阶……}
    \item 不平衡的原因:质量中心与回转中心不平衡\hint{结构不对称、材料不均匀、制造安装
    不准确、零件飞出、磨损、积灰、热变形……}
    \item 两种平衡:静平衡$\sum \vec{F}_i=0$\hint{对$\frac LD\leq\frac15$的盘形件
    最重要}与动平衡$\sum\vec{F}_i=0,\sum\vec{M}_i=0$\hint{对$\frac LD\geq\frac15$
    的细长元件不可忽视};关系\hint{动平衡\then 静平衡}。
    \item 平衡原理:
    \begin{itemize}\tightlist
        \item 静平衡:取$m$及$\vec{r}$\hint{一般控制变量}使$m\vec{r}+\sum m_i
        \vec{r}_i=0$
        \item 动平衡:若$m_1$与$m_2$分布在两平面,分别投影分解至另两个可安装的平面
        \begin{equation}
        \vec{F}_1=\vec{F}'_1+\vec{F}''_1,\quad \vec{F}'_1l'_1=\vec{F}''_1l''_1
        \end{equation}
        使动平衡问题转化为两个静平衡问题。
    \end{itemize}
    \item 平衡试验:
    \begin{itemize}\tightlist
        \item 静平衡:水平导轨上自由滚动\hint{质心在下,上方加重,直至随遇平衡得$mr$}
        \item 动平衡:用动平衡机分析振动信号
        \item \minor{动平衡理论:模态平衡\hint{认为各阶模态相互独立}、影响系数\hint
        {施加不平衡量,由响应推得影响系数}……}
    \end{itemize}
    \item 不平衡量的表示:质径积$m_jr_j$与偏心距$e=\frac{m_jr_j}{m}$;许用不平衡量
    $[e]$\hint{按平衡精度$A$查$[e]=\frac{1000A}\omega(\si{\micro m}$);精度等级记号
    G$A$,ISO标准由G0.4至G4000,对应构件类型}。
    \item \minor{计算:静平衡直接用$[\omega]$,动平衡需分解$[\omega]m$为选定两平面上
    质径积。}
\end{itemize}

\section{螺纹连接}

\subsection{螺纹连接的基本概念}
\begin{itemize}\tightlist
    \item 定义:具有螺纹的零件构成的可拆连接;形成原理\hint{直角三角形绕在圆柱表面}。
    \item 螺纹分类:内/外螺纹、形状\hint{普通螺纹、管螺纹、矩形螺纹、梯形螺纹、锯齿螺纹}
    \minor{、米制/英制螺纹}。
    \item 主要参数:大径$d$\hint{公称直径}、小径$d_1$、中径$d_2$、螺距$P$、线数$n$
    \hint{一般不超过$4$}、导程$S=nP$、导程角$\psi$\hint{以$d_2$为准,可用三角形计算}、
    牙型角$\alpha$、旋向\hint{螺旋旋入的方向,与蜗杆判法相同;右旋占主}。
    \item 普通螺纹:$\alpha=\SI{60}{\degree}$,分粗/细牙\hint{依据螺距大小}。
    \item \minor{管螺纹:$\alpha=\SI{55}{\degree}$,分为圆柱/圆锥管螺纹,无径向间隙
    \hint{用于管路,可以防漏}。}
    \item \minor{其他类型螺纹:自锁性差,传动效率高\hint{可用在丝杠机构等传动装置中}。}
    \item 螺纹连接的类型:螺栓连接\hint{分普通[受拉]螺栓、绞制孔[受剪]螺栓}、螺钉连接
    \hint{不经常拆卸:内螺纹受损}、双头螺柱连接\hint{可经常拆卸:仅螺母受损}、紧定螺钉
    连接\hint{可承受一定载荷};标准螺纹连接件。
\end{itemize}

\subsection{螺纹件的预紧}
\begin{itemize}\tightlist
    \item 预紧原理:装配时适当拧紧螺母/螺钉,被连接件与螺栓均受预紧力$F'$的作用
    \hint{连接件正向受压,螺栓反向受拉},使预紧力矩
    \footnote{公式中,$T_1$与$T_2$分别代表螺纹副内与支撑面-螺母表面上的摩擦阻力矩,
    $r_1\approx\frac{D_1+d_0}4$为螺纹副的当量摩擦角($D_1$及$d_0$分别为螺母环面上的
    外径与内径)。}
    \begin{equation}
    T_\Sigma=T_1+T_2=F'\tan(\psi+\phi_v)\frac{d_2}2+F'f_cr_f
    \end{equation}
    增大\hint{预紧太死,容易使连接件在受冲击时断裂};\minor{测力矩扳手/定力矩扳手的
    使用}。
    \item \emph{预紧是针对受拉螺栓而言的,受剪螺栓不需要预紧!}
    \item 防松措施:
    \begin{itemize}\tightlist
        \item 摩擦防松\hint{施加压力以维持摩擦力,可靠性差}:弹簧垫圈\hint{产生反弹
        力,维持摩擦力}、双螺母\hint{上下产生相反压力使螺栓受拉}。
        \item 机械元件防松\hint{固结螺母与螺栓}:开口销与六角开槽螺母\hint{需要钻孔}、
        圆螺母用止动垫圈\hint{需要开槽}、外舌式止动垫圈\hint{需要紧定螺钉}、串联钢丝
        \hint{用于螺栓组,拆卸不便}。
        \item \emph{串联钢丝的绕法与螺纹旋向有关,钢丝在松动方向上应是受拉的。}
    \end{itemize}
\end{itemize}

\subsection{单个螺栓强度计算}
\begin{itemize}\tightlist
    \item 螺栓载荷形式:轴向力、径向力。
    \item 螺栓断裂情况统计:螺栓根部、螺纹根部、与螺母接合点处最易断裂\hint{应力集中、
    受力不均、结构脆弱……}。
    \item 设计准则:受拉螺栓保证拉伸强度;受剪螺栓保证挤压及剪切强度。
    \item 松螺栓强度:$\sigma=\frac{4F}{\pi d_1^2}\leq[\sigma]$
    \item 仅受预紧力$F'$时强度:按第四强度理论合成预紧力\hint{拉力}与摩擦力矩\hint{剪
    力},得当量应力$\sigma_v\approx1.3\sigma$,以摩擦力$mfF'$抗衡横向载荷$F_R$计算
    $F'$后再确定$\sigma_v$
    \footnote{公式中,$m$为接合面对数(一般为1),$f$为接合面上摩擦因数。}
    :
    \begin{equation}
    mfF'\geq F_R,\quad\sigma_v=\frac{4\times1.3F'}{\pi d_1^2}\leq[\sigma]
    \end{equation}
    \item \minor{用摩擦力抗衡横向载荷并不可靠,常用减载销/套/键等传递横向载荷。}
    \item 受预紧力$F'$及轴向工作拉力$F$时强度:
    \begin{itemize}\tightlist
        \item \emph{螺栓受力不等于$F'+F$}!
        \item 工作拉力$F$使螺栓受拉\hint{变形更大},被连接件受拉\hint{变形回退};$F$
        太大将使连接间产生空隙。
        \item 按相对刚度$\frac{C_1}{C_1+C_2}$\hint{可查表,$C_1$ 为螺栓刚度}合成实际所受轴向力$F_0=
        F'+\frac{C_1}{C_1+C_2}F$与被连接件所受残余锁紧力
        $F''=F'-\frac{C_2}{C_1+C_2}F$。
        \item 静载荷计算:按合成的$F_0$保证$\sigma=\frac{4\times1.3F_0}{\pi d_1^2}
        \leq[\sigma]$。
        \item 疲劳载荷计算:工作拉力为$0\sim F$,则应力幅$\sigma_a=\frac12
        \frac{C_1}{C_1+C_2}\frac{4F}{\pi d_1^2}\leq[\sigma_a]$,$[\sigma_a]=
        \frac{\varepsilon K_mK_u\sigma_{-1}}{[S]_aK_\sigma}$按一般方法查表计算
        \footnote{公式中$K_\sigma$为螺纹制造工艺系数\hint{按$d$查},$K_m$为螺纹制造
        工艺系数\hint{按车制/辗制查},$K_u$为螺纹牙受力不均系数\hint{按受拉/受压查},
        $[S]_a$为安全系数\hint{按预紧状况查},$\sigma_{-1}=0.32\sigma_b$}。
    \end{itemize}
    \item 受剪螺栓强度:在剪力$F_s$下,挤压强度$p=\frac{F_s}{d_0L_{\text{min}}}
    \leq[p]$,剪切强度\footnote{公式中,$d_0$为剪切面直径,$L_\text{min}$为螺纹接触
    段最小长度。}$\tau=\frac{4F_s}{\pi d_0^2}\leq[\tau]$。
    \item 确定$[\sigma]$:$[\sigma]=\frac{\sigma_s}{S}$,安全系数按规格、材料、载荷
    形式、预紧方式确定。
\end{itemize}

\subsection{螺栓组强度计算}
\begin{itemize}\tightlist
    \item 螺栓组载荷形式:轴向载荷、径向载荷、弯矩、转矩。
    \item 结构设计原则:
    \begin{enumerate}\tightlist
        \item 接合面应设计成轴对称的简单形状,并对称布置螺栓。
        \item 受剪螺栓不应大量布置于与平行的方向上\hint{受力不均,没用}。
        \item 承受弯矩及转矩的螺栓组,应离旋转中心尽可能远\hint{并尽量等距}。
        \item 应给扳手留下空间。
        \item 避免在接触面上产生偏心载荷\hint{可做出凸台、沉头座,或加斜面垫圈}。
        \item 合理选择防松装置。
    \end{enumerate}
    \item 螺栓组受力分析:
    \begin{itemize}\tightlist
        \item 受轴向力$F_Q$:平均分担载荷$F=\frac{F_Q}{z}$。
        \item 受横向力$F_R$:受拉螺栓平摊预紧力\hint{以产生摩擦力}$F'=
        \frac{K_fF_R}{fmz}$\hint{其中$K_f>1$为可靠性系数,$m$ 为接合面对数,$z$ 为螺栓数目};
        受剪螺栓平摊剪力$F_s=\frac{F_R}{z}$。
        \item 受转矩$T$:受拉螺栓按回转距$r_i$平摊预紧力$F'=\frac{K_fT}{f(r_1+
        \cdots+r_z)}$;受剪螺栓按变形协调条件$\frac{F_{si}}{r_i}$为常数分配剪力,最大
        剪力$F_{s\max}$出现在最远处:
        \begin{equation}
        F_{s\max}r_{\max}=\frac{r_{\max}^2}{\sum r_i^2}\cdot T
        \end{equation}
        \item 受翻转力矩$M$:按变形协调条件$\frac{F_i}{L_i}=\frac{F_{\max}}{L_{\max}}$
        分担弯曲拉力,最大拉力出现在最远处:
        \begin{equation}
        F_{\max}L_{\max}=\frac{L_{\max}^2}{\sum L_i^2}\cdot M
        \end{equation}
        校核受力最大螺栓时应按相对刚度合成预紧力与弯曲拉力;\minor{还应校核地基接合面:
        \begin{gather}
        p_{\max}\approx\frac{zF'}{A}+\frac{M}{W}\leq[p]\\
        p_{\min}\approx\frac{zF'}{A}-\frac{M}{W}>0
        \end{gather}}
    \end{itemize}
    \item \emph{加高螺母不能提高连接的强度,因受力主要集中在螺纹副接触的开头段。}
    \item 提高连接强度的措施:
    \begin{enumerate}\tightlist
        \item 改善螺纹牙间载荷分配,减少第一圈螺纹牙受力\hint{可采用悬置螺母、内斜螺母、
        环槽螺母,使螺栓螺母变形同步,或平均各段载荷}。
        \item 减小螺栓应力幅\hint{降低螺栓刚度$C_1$,如采用腰状杆螺栓、增大螺栓长度、
        加弹性元件;或提高被连接件刚度$C_2$,如用刚性密封元件替代弹性密封元件}。
        \item 减少应力集中\hint{增大牙根圆角半径,螺纹收尾处设退刀槽等}。
        \item 避免附加完全应力\hint{主要是避免偏心拉}。
        \item 采用合理的制造工艺\hint{特殊工艺处理}。
    \end{enumerate}
\end{itemize}

\section{轴与轴系}
\begin{itemize}\tightlist
    \item 基本作用:支撑回转零件,传递运动与转矩。
    \item 按承载分类:
    \begin{itemize}\tightlist
        \item 心轴:只承受弯矩\hint{不限制轴向转动或不转动,如滑轮主轴}。
        \item 传动轴:只承受转矩\hint{不约束横向转动或不弯曲,如汽车发动机至后桥的轴}。
        \item 转轴:既承受弯矩又承受转矩\hint{对弯曲与转动均有约束,最常见}。
    \end{itemize}
    \item 按形状分类:直轴\hint{光轴及阶梯轴}\minor{、曲轴、挠性轴}。
\end{itemize}

\subsection{轴的强度、刚度、振动强度计算}
\begin{itemize}\tightlist
    \item 轴的失效形式:疲劳断裂、过载断裂、刚度不足\minor{、振幅太大、磨损、蠕变}。
    \item 强度计算:初选轴径或初步校核\hint{可按纯扭转强度或弯扭合成强度计算}。
    \begin{itemize}\tightlist
        \item 按纯扭转强度计算:
        \begin{gather}
        \tau=\frac{T}{W_p}\approx\frac{9550\times10^3\frac Pn}{0.2d^3}\leq[\tau]\\
        d\geq\sqrt[3]{\frac{9550\times10^3}{0.2\times[\tau]}\cdot\frac Pn}=A\times
        \sqrt[3]{\frac Pn}
        \end{gather}
        其中$A$为计算系数\hint{计入了弯矩影响},可根据轴的材料查表。
        \item 按弯扭合成强度计算:作计算简图\hint{轴承为支点},分别作弯矩与转矩图,按第三
        强度理论\footnote{公式中,$\alpha$为折合系数\hint{考虑循环变化},$\delta$为实际
        的循环特征值;对称循环下$\delta=-1$,$\alpha=1$;脉动循环下$\delta=0$,$\alpha
        \approx0.6$;静应力下$\delta=1$,$\alpha\approx0.3$。}
        \begin{equation}
        M_v=\sqrt{M^2+(\alpha T)^2},\quad\alpha=\frac{[\sigma_{-1}]}{[\sigma_\delta]}
        \end{equation}
        合成当量弯矩,再按$\sigma_v=\frac{M_v}{W_z}\leq[\sigma]$校核,此处$[\sigma]$
        在转轴与转动心轴上取$[\sigma_{-1}]$,对其他变化较小情况取$[\sigma_0]$。
        \item \emph{轴上开一个键槽,轴径应增大$3\%\sim5\%$;两个键槽,增大
        $7\%\sim10\%$。}
    \end{itemize}
    \item 刚度校核:
    \begin{itemize}\tightlist
        \item \minor{弯曲刚度:$y\leq[y]$及$\theta\leq[\theta]$\hint{用能量法或当量
        直径计算}。}
        \item 扭转刚度:$\varphi=\frac{584}G\sum\limits_{i=1}^n\dfrac{T_il_i}{d_i^4}\leq
        [\varphi]$\minor{;若含键槽,应乘以刚度降低\hint{变形增大}系数
        $k=\frac1{1-4fh/d}$}。
    \end{itemize}
    \item (弯曲)振动计算:主要考虑一阶临界转速$n_{e1}$,刚性轴要求$n<0.76n_{e1}$。
\end{itemize}

\subsection{轴系设计}
\begin{itemize}\tightlist
    \item 设计要素:材料选择\hint{合用且经济}、结构设计\hint{轴系改错}、工作能力校核。
    \hint{强度、刚度、振动强度}。
    \item 常用材料:碳钢\hint{一般常用}、合金钢\hint{性能好、昂贵,应热处理}、球墨铸铁
    及高强度铸铁\hint{耐磨吸振、易加工、便宜,用于加工形状特殊的轴}\minor{;查表选择}。
    \item 结构设计要素:
    \begin{enumerate}\tightlist
        \item 受力合理、平均,有利于提高强度及刚度。
        \item 定位准确,固定可靠。
        \item 便于加工、装拆、调整。
        \item 减少应力集中,节省材料、减轻质量。
    \end{enumerate}
    \item 轴端划分:轴颈\hint{与轴承配合}、轴头\hint{与齿轮、带轮等一般零件配合}、
    轴肩/轴环\hint{轴向定位用}。
    \item 轴系改错基本原则:固定、定位、可装拆、可靠。
    \item \emph{轴系改错十二条:}
    \begin{enumerate}\tightlist
        \item 轴上各零件应完成周向或轴向固定,不能有多余自由度\hint{常见错误:轴端的
        联轴器、带轮等零件未固定,缺键或轴端挡圈;轴端挡圈或压板直径太小}。
        \item 阶梯轴的阶梯应当是递增或递减
        \footnote{即使相等也不行——特别是有螺纹段与无螺纹段间。}
        的,以便装卸\hint{常见错误:有凸出的轴环,或有下陷段}。
        \item 轴头的长度应略小于与之配合的零件长度\hint{可用退刀槽、砂轮越程槽
        代替},以便可靠地固定\hint{否则无法接触},并利于拆卸。
        \item 固定轴承之轴肩、轴环、套筒、圆螺母的外径应低于轴承内圈,以便用拉马拆卸。
        \item 轴上键槽应打通,使得轴可以自由进出\footnote{键只负责周向定位,在轴向上
        不应受到约束。}。
        \item 正确选择轴承\hint{常见错误:选用单个圆锥、圆柱滚子轴承,不能承受轴向移动;
        成对的轴承方向没有形成O型或X型的对称结构}。
        \item 轴承座与端盖配合处应加工凸台,需要加密封垫圈\hint{用粗黑线表示},并需要
        与轴承外圈配合\footnote{若轴采用一端双向固定,一端游动的结构\hint{如在一侧加用
        套杯},则游动的端盖一侧不应与轴承接触。}。
        \item 轴承配合处不需要键\hint{一般用肩、环等固定,或采用过盈配合}。
        \item 键槽不应太长,以免键的空间位置与其他轴上零件冲突。
        \item 轴承座剖面线应呈上下箱体结构\hint{下面和上面各一系列},以保证同轴度
        \hint{上下拼合}。
        \item 锥面\hint{常来自于齿轮、蜗轮等}不能实现轴向定位。
        \item 多个平键应位于同一母线上\hint{减少装夹时间}。
    \end{enumerate}
    \item 固定件的选择:
    \begin{itemize}\tightlist
        \item 轴端长度:轴头长度比零件轮毂宽度小$1\sim\SI{2}{mm}$;轴颈长度一般等于
        轴承宽度;轴肩或轴环高度$h$取约$0.1d$,圆角半径$r\approx0.75d$,长度$b
        \approx1.4h$。
        \item 轴向固定:轴肩或轴环\hint{简单可靠,应力集中}、套筒\hint{简单可靠,不宜
        过长,可与轴肩、环配合固定两个零件}、圆螺母\hint{装拆方便,应力集中,一般用于
        轴端零件,应采用双螺母或止动垫圈结构}、弹性挡圈\hint{结构简单,应力集中,一般用
        于轴承,常与轴肩配合}、轴端压板\hint{固定轴端零件,需要防转}。
        \item 周向固定:键\hint{(普通)平键、半圆键、楔键、花键,需校核挤压强度}、紧定螺钉
        或销钉\hint{常用于光轴,双向固定,可靠性低}、过盈配合\hint{重压需用温差法或压力
        机装配,可靠性高,对工艺要求高,装拆不便}。
        \item \emph{普通平键的工作面为左右侧面,楔键的工作面为上下表面。}
        \item 轴对机架定位:由轴承实现。
    \end{itemize}
    \item 结构工艺性:结构尽可能简单\hint{如阶梯少、圆角倒角等尺寸统一},螺纹段留出对应
    的槽,加工精度与表面粗糙度合理选择\hint{不宜定得过高}。
    \item 提高强度与刚度的措施:
    \begin{enumerate}\tightlist
        \item 合理布置零件,平均分载:输入零件应布置在轴中部或轴承附近;减少悬臂长度
        \minor{;固结多余的转轴为心轴}。
        \item 改进轴的结构,降低应力集中:用大圆角\minor{、加中心环、开卸荷槽,少开
        螺纹}。
        \item 改善轴的表面质量:减小粗糙度,表面处理。
    \end{enumerate}
    \item \minor{工作能力校核:安全系数法\hint{查表叠加}。}
\end{itemize}

\section{轴承}
\begin{itemize}\tightlist
    \item 轴承的功用:支撑轴系,减小轴与支承间的摩擦、磨损。
    \item 两类轴承:滑动轴承\hint{抗冲击、高精度,用于低速重载、精密仪器、一般支承场合}
    ,滚动轴承\hint{摩擦小、启动快、效率高、已标准化,用于一般机器}。
\end{itemize}

\subsection{滑动轴承}
\begin{itemize}\tightlist
    \item 分类:径向、止推、径向止推;液体润滑\hint{无直接接触}、非液体润滑\hint{有
    微观上直接接触};液体、气体、固体。
    \item 结构组成:轴承座/轴承盖、轴套/轴瓦\hint{与轴接触}、油孔、油槽、紧固件。
    \item 结构形式:整体式\hint{成本低;装拆不便、磨损后不可调隙}、对开式\hint{易调整,
    应用广泛}、自动调心式\hint{补偿径向变形}。
    \item 轴瓦设计\hint{常在表面用轴承合金加工出轴承衬};轴瓦结构形式:整体式、对开式;
    厚壁、薄壁;轴瓦定位:销钉定位、止动螺钉定位、凸缘定位、凸耳定位。
    \item 油孔油槽设计原则:开在油膜压力最小处\hint{一般远离承载区}或轴承剖分面处;油槽
    不能沿轴向完全开通\hint{避免漏油};不允许在轴瓦承载区开槽\hint{不能降低轴承强度和
    油膜承载能力}。
    \item 滑动轴承失效形式:磨粒磨损、粘着磨损\hint{胶合}、疲劳磨损、腐蚀。
    \item 轴承材料选择原则:减摩性、耐磨性、抗胶合性;嵌藏性\hint{容纳磨粒}、顺应性
    \hint{通过弹性变形补偿偏斜}、磨合性;强度、抗腐蚀性;导热性、工艺性、经济性。
    \item 常用轴承材料:轴承合金\hint{硬晶粒+软基体,各方面性能优良,疲劳强度低}、
    铜合金\hint{强度和减摩、耐磨性好,磨合性及嵌藏性差,锡、铝、铅、黄四种各有所长}、
    铸铁\hint{经济,有一定润滑作用,磨合性差,只适用于轻载低速}、多孔金属\hint{自带
    润滑油,但不耐冲击,适用于载荷平稳、加油不便场合}、非金属材料\hint{塑料性能较好,
    不耐热,用于常温轻载;橡胶用于水润场合}。
    \item 工作能力计算:对径向轴承,已知$d,n,F_r$,先根据工况确定结构形式与材料,再分别验算:
    \begin{itemize}\tightlist
        \item 平均压力$p=\frac{F_r}{dB}\leq[p]$\hint{抗挤压强度};
        \item 热载荷$pv=\leq[pv]$\hint{防胶合};
        \item 滑动速度$v\leq[v]$\hint{防止局部$p$及$pv$过大}。
    \end{itemize}
    对止推轴承,也需验算$p$及$pv$,但截面积$A$与径向轴承不同。
    \item \emph{动压润滑条件:}
    \begin{enumerate}\tightlist
        \item 形成收敛楔形间隙\hint{轴必须有偏心距$e$};
        \item 接触面间充满粘性液体;
        \item 接触面间有相对滑动\hint{必须在一定转速下才能发生润滑}。
    \end{enumerate}
    \item 动压润滑与静压润滑区别:初压有无,是否需要供油设备。
    \item \minor{其他类型轴承:动静压润滑轴承、气体轴承、磁力轴承。}
\end{itemize}

\subsection{滚动轴承}
\begin{itemize}\tightlist
    \item 结构:外圈、内圈、滚动体\hint{球或滚子}、保持架。
    \item 分类:按公称接触角$\alpha$,分为向心轴承\hint{$\alpha\leq\SI{45}{\degree%
    }$,承受径向载荷}与推力轴承\hint{$\alpha>\SI{45}{\degree}$,承受轴向载荷}。
    \item 主要性能:承载能力\hint{滚子优于球$1.5\sim3$倍}、极限转速、允许角偏差\hint%
    {能否调心}。
    \item 滚动轴承类别:
    \begin{itemize}\tightlist
        \item 调心球轴承\hint{1}:径向为主,可调心;
        \item 调心滚子轴承\hint{2}:径向为主,可调心,可承受重载;
        \item 圆锥滚子轴承\hint{3}:径向轴向皆可承受,外圈可分离,成对使用;
        \item 推力球轴承\hint{5}:只受轴向,不耐高速;
        \item 深沟球轴承\hint{6}:径向为主,最常用,价格最低;
        \item 角接触球轴承\hint{7}:径向轴向皆可承受,$\alpha$可取\SI{15}{\degree}
        \hint{C}、\SI{25}{\degree}\hint{AC}、\SI{40}{\degree}\hint{B}三种,成对
        使用;
        \item 圆柱滚子轴承\hint{N}:只受径向,外圈可分离,不可调心,耐冲击。
    \end{itemize}
    \item 轴承(基本)代号:类型代号-[宽度系列代号-]直径系列代号-内径代号\minor{%
    [-内部结构代号-公差等级代号-游隙代号]}
    \begin{itemize}\tightlist
        \item 类型代号:见「滚动轴承类别」。
        \item 宽度系列代号:\minor{系列为$8012345$,}一般为$0$,除调心轴承与圆锥滚子
        轴承外均可省略。
        \item 直径系列代号:\minor{系列为$789012345$,}不能省略。
        \item 内径代号:$00-03$分别对应$10,12,15,\SI{17}{mm}$,$04-99$乘五即得对应
        直径\hint{mm},超过$\SI{500}{mm}$的用分母形式表示。
        \item 内部结构代号:如角接触轴承的三种$\alpha$。
        \item 公差等级代号:\minor{系列为$0(6x)6542$,}一般为$0$并省略\minor{,否则
        应标出``/P$n$''}。
        \item 游隙代号:\minor{系列为$120345$},一般为$0$并省略\minor{,否则应标出
        ``/C$n$''}。
    \end{itemize}
    \item 滚动轴承的选择依据:
    \begin{enumerate}\tightlist
        \item 轻载或高速宜用球轴承,冲击或重载宜用滚子轴承;能用球轴承时不要用滚子轴承
        \hint{经济性}。
        \item 纯径向载荷宜用向心轴承,纯轴向宜用推力轴承;同时承受径向与轴向载荷,需结
        合两者相对大小具体考虑。
        \item 考虑调心需求\hint{此时避免用圆柱滚子轴承}。
        \item 整体轴承座中的轴承,优先考虑外圈可分离的\hint{便于拆卸}。
    \end{enumerate}
    \item 滚动轴承的载荷情况:
    \begin{enumerate}\tightlist
        \item 理论承载区为整个下半部分,实际承载区小于一半;附加轴向载荷有助于扩大承
        载区\hint{预紧}。
        \item $\alpha>0$时,总有轴向附加力$S=F_R\tan\alpha$\hint{应成对使用}。
        \item 元件所受载荷总呈脉动循环:外圈是稳定脉动循环,内圈不稳定,滚动体上循环
        频率最高\hint{公转+自转}。
    \end{enumerate}
    \item 常见失效形式:疲劳点蚀\hint{主要,脉动循环导致};磨粒磨损、屈服失效;轴承故障
    \hint{人为因素导致}。
    \item 寿命校核基本概念:基本额定寿命$L_{10}$\hint{确定条件下,失效率达10\%}、基本
    额定寿命载荷$C$\hint{确定条件下使$L_{10}$达$10^6$转}、基本额定静载荷$C_0$\hint{确
    定条件下使接触应力达一定值}。
    \item 动载寿命校核:用$P=(XF_r+YF_a)f_p$折算实际载荷 $(F_r,F_a)$ 为当量载荷$P$,用公式
    \footnote{公式中,$\varepsilon$为寿命指数,对球轴承取$3$,滚子轴承取$10/3$。}
    \begin{equation}
        L_{10}=\left(\frac CP\right)^\varepsilon\times10^6
    \end{equation}
    核算寿命\hint{为多少转,可进一步换算为小时数}。
    \item \minor{可查表将已算好的$L_{10}$转化为其他的$L_n$。}
    \item \minor{静载校核条件:$C_0\geq S_0P$,$S_0$查表,$P$需根据轴承类型合成、折
    算。}
    \item 角接触球轴承的轴向载荷计算:
    \begin{itemize}\tightlist
        \item 装法:正装\hint{X型,接触法线向内,轴向附加力对内},反装\hint{O型,
        接触法线向外,轴向附加力对外}。
        \item 计算原理:先做受力简图,理论上应有
        \begin{equation}\label{eq:轴向平衡条件}
        F_a+S_1=S_2
        \end{equation}
        实际上$F_a$往往偏大或偏小,此时应在
        方程一端附加支反力$S_1'$或$S_2'$\hint{来自于支承接触面}以维持等号。
        \item \emph{$S_1$与$S_2$由径向载荷决定,不能改变,可变的是附加支反力!}
        \item 做校核时,实际轴向力$F_a$取方程(\ref{eq:轴向平衡条件})两侧之最大值。
    \end{itemize}
    \item 轴承支承方式:两端单向固定;一端双向固定、一端游动\hint{留出自由伸缩余地};
    两端游动\hint{便于调整,如自适应的人字齿}。
    \item 利用调整垫片可控制游隙。
    \item \emph{轴承与轴的配合用基孔制\hint{紧},与座孔的配合用基轴制\hint{紧}。}
    \item 轴承固定:内圈——轴肩、弹性挡圈、轴端压板、圆螺母-止动垫圈、锥形套筒;
    外圈——孔用挡肩、轴承端盖、弹性挡圈、外圈带止动槽-止动环\hint{用法请仔细阅读课本}。
    \item \minor{轴承是专门生产的标准件,不允许加工。}
    \item 轴承装拆原则:不允许通过滚动体传力\hint{不能使内外圈错位}。
    \item 轴承预紧:针对成对轴承较为有效;分定位预紧\hint{「形锁合」}与定压预紧\hint{%
    「力锁合」}。
    \item 润滑与密封:油润滑与脂润滑;\minor{润滑方式包括油浴、滴油、飞溅、喷油、油雾、
    油气等;}密封分接触式\hint{毡圈、唇型密封圈}与非接触式\hint{间隙式、迷宫式}。
\end{itemize} 

\section{联轴器与离合器}
\begin{itemize}\tightlist
    \item 共性:连接两轴及轴上回转件,传递运动和转矩。。
    \item \emph{区别:联轴器只能在停车时才能拆开;离合器可在工作中接合或分离。}
    \item 联轴器的特殊要求:补偿轴的偏移\hint{能补偿者为挠性,否则为刚性};吸振缓冲。
    \item 刚性联轴器:套筒联轴器\hint{用锥销或平键连接},凸缘联轴器\hint{配合对中,或
    用受剪螺钉固定}。
    \item 挠性联轴器:分无弹性元件\hint{利用可动元件配合}和有弹性元件两种。
    \begin{itemize}\tightlist
        \item 无弹性元件:十字滑块联轴器、齿轮联轴器\hint{耐重载}、万向联轴器
        \hint{轴线相交}。
        \item 有弹性元件\hint{高速、冲击适用}:弹性套柱销联轴器、弹性柱销联轴器。
    \end{itemize}
    \item 联轴器的选择:先初选类别,查手册,校核转矩$T_c=KT\leq[T]$\hint{含工况系数}
    及转速$n\leq[n]$。
    \item 离合器类别:
    \begin{itemize}\tightlist
        \item 嵌合式离合器:同步回转,工作可靠,有刚冲。实例:牙嵌式、齿轮式。
        \item 摩擦式离合器:无级变速,工作平稳,不同步。实例:单盘式、多盘式。
        \item \minor{超越式离合器:单方向传矩。实例:滚珠式、棘轮式\hint{见「棘轮机构」
        章节}。}
        \item \minor{安全离合器:转矩过大时分离或打滑,可保护机器。}
    \end{itemize}
\end{itemize}

\end{document}