\documentclass[12pt,a4paper]{article}
\usepackage[margin=1in]{geometry}

\usepackage[heading,hyperref]{ctex}
\usepackage{amsmath}
\usepackage{unicode-math}
\setCJKmainfont{思源宋体}
\setCJKsansfont{思源黑体}
\setmainfont{Cambria}
\setsansfont{Calibri}
\setmathfont{Cambria Math}

\usepackage{fancyhdr}
\usepackage{xcolor}
\lhead{}\rhead{}
\chead{\textsc{Outline on Fundamentals of Mechanical Design}}
\pagestyle{fancy}
\setcounter{secnumdepth}{0}
\setlength{\headheight}{15pt}

\usepackage{siunitx}
\usepackage{fontawesome5}

\newcommand{\tightlist}{\setlength{\parskip}{0pt}\setlength{\itemsep}{0pt}}
\newcommand{\hint}[1]{\textsf{(#1)}}
\newcommand{\minor}[1]{{\color{gray} #1}}
\newcommand{\then}{$\to$}
\newcommand{\beginday}{2019 年 6 月 30 日}
\renewcommand{\emph}[1]{\faIcon[regular]{lightbulb}\ \textbf{#1}}

\title{《机械设计基础》考点提纲}
\date{\beginday--\today}
\author{黑山雁}

\begin{document}
\maketitle

\section{绪论与概述}
\subsection{基本概念}
\begin{itemize}\tightlist
    \item 机器:执行机械运动的装置\hint{人为实体组合、确定相对运动、功能转换或做有用功}。
    \item 机构:只有运动变换,没有能量转换;机器的运动部分\hint{剔除了与运动无关的因素}。
    \item 构件:运动的单元体;零件:加工制造的单元体\hint{注意区分}。
    \item \minor{机械:机器与机构的总称。}
\end{itemize}

\subsection{设计原则}
\begin{itemize}\tightlist
    \item 传统设计方法:依靠基础物理与经验汇总;现代设计方法:计算机辅助、优化、可靠性
    设计。
    \item \hint{零件}失效:不能在预定条件、预定期限下正常工作;工作能力:抵抗失效的能力。
    \item \minor{功能分析:总功能\then 功能元\then 功能载体\hint{机构、零件};原理分析
    :形态学矩阵,择优选取。}
    \item 机械设计基本要求:
    \begin{enumerate}\tightlist
        \item 功能要求:实现预定功能,不失效。
        \item 可靠性要求:减少零件数目。\emph{零件越多,可靠度越低。}
        \item 经济性要求:能用便宜的,就不用贵的。
        \item \minor{操作方便和安全要求。}
        \item \minor{造型、色彩要求。}
    \end{enumerate}
    \item 零件设计一般程序\hint{之后各章遵循的逻辑}:
    \begin{enumerate}\tightlist
        \item 初选结构形式、选择零件类型\hint{读型号,分析需求,结合零件特点查表选型}。
        \item 计算作用在零件上的载荷\hint{受力、应力分析}。
        \item 选择零件材料及热处理方式\hint{“工程材料基础”、“金工实习”}。
        \item 零件工作能力设计\hint{强度、刚度、疲劳、振动、热平衡}。
        \item 零件结构设计\hint{尺寸、形状}。
        \item 校核计算\hint{确定一个,核另一个}。
        \item \minor{绘制零件工作图、编写设计计算说明书}。
    \end{enumerate}
    \item 三种设计:开发性设计\hint{创新}、适应性设计\hint{改进}、变型设计\hint{改参数}。
\end{itemize}

\section{运动设计基础}

\section{工作能力与结构设计基础}

\section{平面连杆机构}

\section{凸轮机构}

\section{齿轮传动}

\section{蜗杆传动}

\section{轮系}

\section{带传动}
\subsection{总述}
\begin{itemize}\tightlist
    \item 带传动的三个组成部分;刚性传动与挠性传动;两类带传动。
    \item 三类主要传动形式\hint{及适用的速度范围};四种带\hint{形状、实际工作面、优缺点}。
    \item 当量摩擦因数\hint{谁的?计算公式、含义}。
    \item 五个主要几何参数、它们的关系\hint{近似计算公式}。
    \item 带传动的六个特点\hint{减震、打滑、效率、结构与寿命、尺寸、传动比};应用范围
    \hint{功率、带速}。
\end{itemize}

\subsection{V带与带轮}
\begin{itemize}\tightlist
    \item 三种V带;V带的组成\hint{四个部分}。
    \item 普通V带的参数\hint{型号——四项几何尺寸};轮槽楔角\hint{与V带楔角的关系}。
    \item 带的节面\hint{定义}与带轮基准直径。
    \item \minor{另两种V带的应用场合\hint{功率、尺寸、工况}。}
    \item \minor{带轮的设计要求\hint{质量、安装、平衡性、表面};}带轮材料
    \hint{与速度关系};\minor{四种结构形式\hint{及选用依据}。}
\end{itemize}

\subsection{工作原理}
\begin{itemize}\tightlist
    \item 基本工作原理\hint{带轮如何驱动};二力相抗原理\hint{哪两个力?随载荷加大的变
    化趋势};带传动设计的主要问题。
    \item 张紧力、初拉力$F_0$;紧边与松边\hint{识别、形成原理};紧边拉力$F_1$、松边
    拉力$F_2$\hint{方向、大小关系};不变形条件\footnote{经常被遗忘。}
    \hint{$F_0$、$F_1$、$F_2$的等式};有效工作拉力$F_e$、摩擦力$F'$\hint{关系、平衡
    原理、计算公式};打滑原理\hint{谁顶不住了?}。
    \item \emph{一般总是小带轮发生打滑,其包角应尽可能的大。}
    \item 欧拉公式\hint{$F_1=F_2\mathrm{e}^{f\alpha}$,说了什么?};最大有效工作
    拉力$F_{ec}$\hint{计算公式,从$F_1$到$F_0$};打滑位点\hint{大或小?}。
    \item 避免打滑的三点原则\hint{来自于$F_{ec}$的计算公式};\minor{为什么铸铁优于
    钢。}
    \item 三种应力:拉应力\hint{估算方法};弯曲应力\hint{估算方法、谁大?};离心拉应力
    \hint{均匀、估算方法};应力分布规律\hint{绘出示意图};应力最大点;带的疲劳强度条件。
    \item 弹性滑动\hint{形成原理、方向};速度不等式$v_1>v_\text{带}>v_2$\hint{为什
    么?};滑动率\hint{计算公式、大致数值};传动比不恒定\hint{计算公式};弹性滑动的
    影响\hint{效率、发热、速度}。
    \item 滑动弧、静弧;打滑与滑动弧的关系;打滑的危害\hint{磨损、发热、速度、传动}。
\end{itemize}

\subsection{校核与设计}
\begin{itemize}\tightlist
    \item 四种失效形式\hint{打滑、疲劳、磨损、\minor{带轮断裂}};主要设计依据
    \hint{“一个保证,一个足够”}。
    \item 强度计算方法\hint{基本条件、$[\sigma]$条件\then 关于$F_e$的条件\then 功率
    条件};最佳带速\hint{功率随速度的变化规律};基本额定功率$P_0$\hint{条件、表格构成}。
    \item 一般条件下的功率校核:额定功率$[P]=(P_0+\Delta P_0)K_\alpha K_L$,其中
    $\Delta P_0=K_bn_1(1-1/K_i)$\hint{各项含义、随相关参数的变化规律、表格构成}。
    \item 设计计算步骤:
    \begin{enumerate}\tightlist
        \item 条件\hint{工况、带速或传动比、功率};
        \item 计算功率$P_d=K_AP$\hint{含义};
        \item 初选型号\hint{查图};
        \item 确定带轮基准直径\hint{小带轮查表过限,大带轮按传动比圆整};
        \item 验算带速\hint{多少为宜};
        \item 确定$a$及$L_d$\hint{不等式初选$a$,估算$L_d$并查表圆整,再近似反算$a$};
        \item 验算小带轮包角\hint{公式;一般要求};
        \item 确定根数$z$\hint{计算载荷、额定功率};
        \item 计算初拉力$F_0$\hint{公式};
        \item 计算压轴力$F_P$\hint{受力分析,导出公式}。
    \end{enumerate}
    \item \minor{张紧装置设计\hint{滑道、摆架、重力自动张紧、张紧轮}。}
    \item 使用与维护原则\hint{同心平行对中;带在轮槽中的位置;成组时尺寸均匀……}。
\end{itemize}

\section{间歇传动机构}
\subsection{棘轮机构}
\begin{itemize}\tightlist
    \item 组成部分\hint{摇杆、棘爪、棘轮、止回爪、机架};运动原理\hint{摆动“拖刀”};
    转换方向\hint{摆动\then 间歇转动}。
    \item 分类:齿式/摩擦式、单向/可换向、外啮合/内啮合、齿式/双动式。
    \item 特点:
    \begin{itemize}\tightlist
        \item 齿式:工作可靠,精度高;有硬冲,噪声、磨损严重。
        \item 摩擦式:工作平稳,无级调整;精度差。
    \end{itemize}
    \item 应用场合:转位分度与送进\hint{牛头刨刀}、止动\hint{起重}、超越离合器\hint{
    自行车后轴——原理}。
    \item 齿式棘轮机构设计:模数$m=\frac{d_a}{z}$\hint{基准在齿顶圆上},齿数$z$
    \hint{按分度确定,一般取$8\sim60$},轮齿倾斜角$\alpha$\hint{条件$\alpha>\varphi$
    ,一般取$\alpha=\SI{20}{\degree}$}。
\end{itemize}

\subsection{槽轮机构}
\begin{itemize}\tightlist
    \item 组成部分\hint{拨盘、槽轮};运动原理;转换方向\hint{连续转动\then 间歇不等速
    转动}。
    \item 分类:外槽轮/内槽轮\hint{记住形状}。
    \item 特点:工作可靠且较平稳,效率高;有柔冲,分度难调整\hint{受工艺限制,槽数不宜
    过多}。
    \item 运动分析与设计准则:
    \begin{itemize}\tightlist
        \item 运动系数:$k=\frac{\text{运动时间}}{\text{总时间}}$,外槽轮$k=\frac12
        -\frac1z<0.5$,内槽轮$k=\frac12+\frac1z>0.5$。
        \item 意义:若要使外槽轮$k>0.5$,应采用$n>1$个拨销使$k=n\left(\frac12-\frac%
        1z\right)>0.5$\hint{但不能超过$1$,由此可解出$n$的最大值};增大$z$也可适当增加
        运动系数;单拨销外槽轮的$z>3$才能正常运转\hint{否则$k\leq0$}。
        \item 齿数设计:一般取$z=4,6,8$。
        \item \minor{运动特性:槽轮转动非等速;角速度与角加速度取决于$z$。}
    \end{itemize}
\end{itemize}

\subsection{不完全齿轮机构}
\begin{itemize}\tightlist
    \item 组成部分\hint{两轮;凹/凸锁止弧};运动原理\hint{有齿时啮合传动,无齿时锁止};
    转换方向\hint{连续转动\then 间歇等速转动}。
    \item 分类:外啮合、内啮合。
    \item 特点:结构简单、分度设计灵活、精度高;有硬冲\hint{继承于齿轮,可用附加杆装置
    减缓冲击}。
\end{itemize}


\section{机械系统动力学}
\begin{itemize}\tightlist
    \item 三大问题:机械真实运动;飞轮调节波动;转子平衡问题。
    \item \minor{原理部分:设计的灵魂。}
    \item \minor{现代设计方法:以动力学设计囊括静态设计。}
\end{itemize}

\subsection{机械真实运动分析}
\begin{itemize}\tightlist
    \item 作用在机械上的力:\minor{重力、惯性力、约束反力、摩擦力、}驱动力、工作阻力
    \footnote{摩擦力并不被视为工作阻力,而被视为“有害阻力”。}
    \item 力的机械特性:力--运动参数\hint{机械特性曲线}
    \item 常见机械特性:
    \begin{itemize}\tightlist
        \item 驱动力:常数\hint{液压活塞}、位置函数\hint{内燃机}、速度函数\hint{电动机}
        \item 工作阻力:常数\hint{起重机}、位置函数\hint{压缩机}、速度函数\hint%
        {鼓风机、离心泵}、时间函数\hint{破碎机}。
        \item \minor{特性的确定:其他课程中解决。}
    \end{itemize}
    \item 机械运动的阶段\hint{三个}:启动、稳定运动\hint{主动件保持常速或周期波动——
    匀速稳定运动、变速稳定运动}、停车;正常工作速度
    \item 功能分析:$W_a-W_c=\frac12\Delta\sum(m_iv_{c_i}^2+J_{c_i}\omega^2)$;
    分阶段情况\hint{$W_a-W_c>0$\then$=0$\then$<0$}。
    \item 等效动力学模型:等效构件、等效力\hint{或力矩}、等效质量\hint{或转动惯量};
    分析方法:功率相等原则、动能相等原则;其他事项\hint{常将$F_a$与
    $F_c$分开分析}。
    \begin{gather}
    \left\{
    \begin{aligned}
    F_\text{v}&=\sum_iF_i\frac{v_i\cos\alpha_i}{v}+\sum_i\left(\pm M_i\frac%
    {\omega_i}{v}\right)\\
    m_\text{v}&=\sum_i\left[m_i\left(\frac{v_{c_i}}{v}\right)^2+J_{c_i}\left(
    \frac{\omega_i}{v}\right)^2\right]
    \end{aligned}\right.\\\left\{
    \begin{aligned}
    M_\text{v}&=\sum_iF_i\frac{v_i\cos\alpha_i}{\omega}+\sum_i\left(\pm M_i
    \frac{\omega_i}{\omega}\right)\\
    J_\text{v}&=\sum_i\left[m_i\left(\frac{v_{c_i}}{\omega}\right)^2+J_{c_i}\left(
    \frac{\omega_i}{\omega}\right)^2\right]
    \end{aligned}\right.
    \end{gather}
    \item 等效分析的结果\hint{速度比的函数,仅与位置有关;不等于简单的求和\minor{,除非
    ……}}。
    \item 等效分析的意义:使系统仅剩一个自由度,便于求解。
    \begin{equation}
    F_\text{v}=m_v\frac{\mathrm{d}v}{\mathrm{d}t}
    \end{equation}
\end{itemize}

\subsection{速度波动与飞轮设计}
\begin{itemize}\tightlist
    \item 调速原理:增大构件的质量或转动惯量\then 安装飞轮\hint{起水库之用};飞轮的
    影响\hint{降低速度波动幅值、减少原动机输出功率;延长启动与制动时间}。
    \item \emph{一般将飞轮装在高速轴上,可减轻飞轮的质量及尺寸。}
    \item 平均角速度$\omega_m$:实际平均角速度\hint{积分平均}与算术平均角速度;不均匀
    系数$\delta=\frac{\omega_\text{max}-\omega_\text{min}}{\omega_\text{m}}$及
    许用不均匀系数$[\delta]$\hint{大致量级;利用$[\delta]$反求速度波动许可值}
    \item 飞轮设计原理:计算$J_F\approx J_\text{v}$\hint{飞轮比机械自身转动惯量大
    很多},保证由
    \begin{equation}
    W_y=\int_{\varphi_{E_\text{min}}}^{\varphi_{E_\text{max}}}M_\text{v}\,
    \mathrm{d}\varphi=\frac12J_F(\omega_\text{max}^2-\omega_\text{min}^2)
    =\frac12J_F\omega_m^2\delta
    \end{equation}
    所确定的$\delta\leq[\delta]$,故应有
    \begin{equation}
    J_F=\frac{W_y}{\omega^2\delta}.
    \end{equation}
    \item 最大盈亏功$W_y$的确定:$M_\text{v}-\varphi$图
    \footnote{一般为$M_\text{va}$与$M_\text{vc}$,此时应求两曲线所夹面积(即作差)。}
    \then 能量指示图\hint{曲线包围面积}\then 最大盈亏功\hint{最值点之差}
    \item \minor{非周期性速度波动:调速器}
\end{itemize}

\subsection{转子的动平衡与静平衡}
\begin{itemize}\tightlist
    \item 平衡问题的意义:减小振动\hint{包括共振}造成的危害;三类平衡问题\minor{
    \hint{刚性转子、挠性转子、机械在基座上的平衡}}。
    \item 临界转速$n_c$\hint{第一阶,第二阶……}
    \item 不平衡的原因:质量中心与回转中心不平衡\hint{结构不对称、材料不均匀、制造安装
    不准确、零件飞出、磨损、积灰、热变形……}
    \item 两种平衡:静平衡$\sum \vec{F}_i=0$\hint{对$\frac LD\leq\frac15$的盘形件
    最重要}与动平衡$\sum\vec{F}_i=0,\sum\vec{M}_i=0$\hint{对$\frac LD\geq\frac15$
    的细长元件不可忽视};关系\hint{动平衡\then 静平衡}。
    \item 平衡原理:
    \begin{itemize}\tightlist
        \item 静平衡:取$m$及$\vec{r}$\hint{一般控制变量}使$m\vec{r}+\sum_im_i
        \vec{r}_i=0$
        \item 动平衡:若$m_1$与$m_2$分布在两平面,分别投影分解至另两个可安装的平面
        \begin{equation}
        \vec{F}_1=\vec{F}'_1+\vec{F}''_1,\quad \vec{F}'_1l'_1=\vec{F}''_1l''_1
        \end{equation}
        使动平衡问题转化为两个静平衡问题。
    \end{itemize}
    \item 平衡试验:
    \begin{itemize}\tightlist
        \item 静平衡:水平导轨上自由滚动\hint{质心在下,上方加重,直至随遇平衡得$mr$}
        \item 动平衡:用动平衡机分析振动信号
        \item \minor{动平衡理论:模态平衡\hint{认为各阶模态相互独立}、影响系数\hint
        {施加不平衡量,由响应推得影响系数}……}
    \end{itemize}
    \item 不平衡量的表示:质径积$m_jr_j$与偏心距$e=\frac{m_jr_j}{m}$;许用不平衡量
    $[e]$\hint{按平衡精度$A$查$[e]=\frac{1000A}\omega(\si{\micro m}$);精度等级记号
    G$A$,ISO标准由G0.4至G4000,对应构件类型}。
    \item \minor{计算:静平衡直接用$[\omega]$,动平衡需分解$[\omega]m$为选定两平面上
    质径积。}
\end{itemize}

\section{螺纹连接}

\subsection{螺纹连接的基本概念}
\begin{itemize}\tightlist
    \item 定义:具有螺纹的零件构成的可拆连接;形成原理\hint{直角三角形绕在圆柱表面}。
    \item 螺纹分类:内/外螺纹、形状\hint{普通螺纹、管螺纹、矩形螺纹、梯形螺纹、锯齿螺纹}
    \minor{、米制/英制螺纹}。
    \item 普通螺纹:$\alpha=\SI{60}{\degree}$,$\phi_v$大,分粗/细牙\hint{依据螺距
    大小}。
    \item \minor{管螺纹:$\alpha=\SI{55}{\degree}$,分为圆柱/圆锥管螺纹,无径向间隙
    \hint{用于管路,可以防漏}。}
    \item \minor{其他类型螺纹:自锁性差,传动效率高\hint{可用在丝杠机构等传动装置中}。}
    \item 主要参数:大径$d$\hint{公称直径}、小径$d_1$、中径$d_2$、螺距$P$、线数$n$
    \hint{一般不超过$4$}、导程$S=nP$、导程角$\psi$\hint{以$d_2$为准,可用三角形计算}、
    牙型角$\alpha$、旋向\hint{螺旋旋入的方向,与蜗杆判法相同;右旋占主}。
    \item 螺纹连接的类型:螺栓连接\hint{分普通[受拉]螺栓、绞制孔[受剪]螺栓}、螺钉连接
    \hint{不经常拆卸:内螺纹受损}、双头螺柱连接\hint{可经常拆卸:仅螺母受损}、紧定螺钉
    连接\hint{可承受一定载荷};标准螺纹连接件。
\end{itemize}

\subsection{螺纹件的预紧}
\begin{itemize}\tightlist
    \item 预紧原理:装配时适当拧紧螺母/螺钉,被连接件与螺栓均受预紧力$F'$的作用
    \hint{连接件正向受压,螺栓反向受拉},使预紧力矩
    \footnote{公式中,$T_1$与$T_2$分别代表螺纹副内与支撑面-螺母表面上的摩擦阻力矩,
    $r_1\approx\frac{D_1+d_0}4$为螺纹副的当量摩擦角($D_1$及$d_0$分别为螺母环面上的
    外径与内径)。}
    \begin{equation}
    T_\Sigma=T_1+T_2=F'\tan(\psi+\phi_v)\frac{d_2}2+F'f_cr_f
    \end{equation}
    增大\hint{预紧太死,容易使连接件在受冲击时断裂};\minor{测力矩扳手/定力矩扳手的
    使用}。
    \item \emph{预紧是针对受拉螺栓而言的,受剪螺栓不需要预紧!}
    \item 防松措施:
    \begin{itemize}\tightlist
        \item 摩擦防松\hint{施加压力以维持摩擦力,可靠性差}:弹簧垫圈\hint{产生反弹
        力,维持摩擦力}、双螺母\hint{上下产生相反压力使螺栓受拉}。
        \item 机械元件防松\hint{固结螺母与螺栓}:开口销与六角开槽螺母\hint{需要钻孔}、
        圆螺母用止动垫圈\hint{需要开槽}、外舌式止动垫圈\hint{需要紧定螺钉}、串联钢丝
        \hint{用于螺栓组,拆卸不便}。
        \item \emph{串联钢丝的绕法与螺纹旋向有关,钢丝在松动方向上应是受拉的。}
    \end{itemize}
\end{itemize}

\subsection{单个螺栓强度计算}
\begin{itemize}\tightlist
    \item 螺栓载荷形式:轴向力、径向力。
    \item 螺栓断裂情况统计:螺栓根部、螺纹根部、与螺母接合点处最易断裂\hint{应力集中、
    受力不均、结构脆弱……}。
    \item 设计准则:受拉螺栓保证拉伸强度;受剪螺栓保证挤压及剪切强度。
    \item 松螺栓强度:$\sigma=\frac{4F}{\pi d_1^2}\leq[\sigma]$
    \item 仅受预紧力$F'$时强度:按第四强度理论合成预紧力\hint{拉力}与摩擦力矩\hint{剪
    力},得当量应力$\sigma_v\approx1.3\sigma$,以摩擦力$mfF'$抗衡横向载荷$F_R$计算
    $F'$后再确定$\sigma_v$
    \footnote{公式中,$m$为接合面对数(一般为1),$f$为接合面上摩擦因数。}
    :
    \begin{equation}
    mfF'\geq F_R,\quad\sigma_v=\frac{4\times1.3F'}{\pi d_1^2}\leq[\sigma]
    \end{equation}
    \item \minor{用摩擦力抗衡横向载荷并不可靠,常用减载销/套/键等传递横向载荷。}
    \item 受预紧力$F'$及轴向工作拉力$F$时强度:
    \begin{itemize}\tightlist
        \item \emph{螺栓受力不等于$F'+F$}!
        \item 工作拉力$F$使螺栓受拉\hint{变形更大},被连接件受拉\hint{变形回退};$F$
        太大将使连接间产生空隙。
        \item 按相对刚度$\frac{C_1}{C_1+C_2}$\hint{可查表}合成实际所受轴向力$F_0=
        F'+\frac{C_1}{C_1+C_2}F$与被连接件所受残余锁紧力
        $F''=F'-\frac{C_2}{C_1+C_2}F$。
        \item 静载荷计算:按合成的$F_0$保证$\sigma=\frac{4\times1.3F_0}{\pi d_1^2}
        \leq[\sigma]$。
        \item 疲劳载荷计算:工作拉力为$0\sim F$,则应力幅$\sigma_a=\frac12
        \frac{C_1}{C_1+C_2}\frac{4F}{\pi d_1^2}\leq[\sigma_a]$,$[\sigma_a]=
        \frac{\varepsilon K_mK_u\sigma_{-1}。}{[S]_aK_\sigma}$按一般方法查表计算
        \footnote{公式中$K_\sigma$为螺纹制造工艺系数\hint{按$d$查},$K_m$为螺纹制造
        工艺系数\hint{按车制/辗制查},$K_u$为螺纹牙受力不均系数\hint{按受拉/受压查},
        $[S]_a$为安全系数\hint{按预紧状况查},$\sigma_{-1}=0.32\sigma_b$}。
    \end{itemize}
    \item 受剪螺栓强度:在剪力$F_s$下,挤压强度$p=\frac{F_s}{d_0L_{\text{min}}}
    \leq[p]$,剪切强度\footnote{公式中,$d_0$为剪切面直径,$L_\text{min}$为螺纹接触
    段最小长度。}$\tau=\frac{4F_s}{\pi d_0^2}\leq[\tau]$。
    \item 确定$[\sigma]$:$[\sigma]=\frac{\sigma_s}{S}$,安全系数按规格、材料、载荷
    形式、预紧方式确定。
\end{itemize}

\subsection{螺栓组强度计算}
\begin{itemize}\tightlist
    \item 螺栓组载荷形式:轴向载荷、径向载荷、弯矩、转矩。
    \item 结构设计原则:
    \begin{enumerate}\tightlist
        \item 接合面应设计成轴对称的简单形状,并对称布置螺栓。
        \item 受剪螺栓不应大量布置于与平行的方向上\hint{受力不均,没用}。
        \item 承受弯矩及转矩的螺栓组,应离旋转中心尽可能远\hint{并尽量等距}。
        \item 应给扳手留下空间。
        \item 避免在接触面上产生偏心载荷\hint{可做出凸台、沉头座,或加斜面垫圈}。
        \item 合理选择防松装置。
    \end{enumerate}
    \item 螺栓组受力分析:
    \begin{itemize}\tightlist
        \item 受轴向力$F_Q$:平均分担载荷$F=\frac{F_Q}{z}$。
        \item 受横向力$F_R$:受拉螺栓平摊预紧力\hint{以产生摩擦力}$F'=
        \frac{K_fF_R}{fmz}$\hint{其中$K_f>1$为可靠性系数};受剪螺栓平摊剪力$F_s=
        \frac{F_R}{z}$。
        \item 受转矩$T$:受拉螺栓按回转距$r_i$平摊预紧力$F'=\frac{K_fT}{f(r_1+
        \cdots+r_z)}$;受剪螺栓按变形协调条件$\frac{F_{si}}{r_i}$为常数分配剪力,最大
        剪力$F_{s\max}$出现在最远处:
        \begin{equation}
        F_{s\max}r_{\max}=\frac{r_{\max}^2}{\sum r_i^2}\cdot T
        \end{equation}
        \item 受翻转矩$M$:按变形协调条件$\frac{F_i}{L_i}=\frac{F_{\max}}{L_{\max}}$
        分担弯曲拉力,最大拉力出现在最远处:
        \begin{equation}
        F_{\max}L_{\max}=\frac{L_{\max}^2}{\sum L_i^2}\cdot M
        \end{equation}
        校核受力最大螺栓时应按相对刚度合成预紧力与弯曲拉力;\minor{还应校核地基接合面:
        \begin{gather}
        p_{\max}\approx\frac{zF'}{A}+\frac{M}{W}\leq[p]\\
        p_{\min}\approx\frac{zF'}{A}-\frac{M}{W}>0
        \end{gather}}
    \end{itemize}
    \item \emph{加高螺母不能提高连接的强度,因受力主要集中在螺纹副接触的开头段。}
    \item 提高连接强度的措施:
    \begin{enumerate}\tightlist
        \item 改善螺纹牙间载荷分配,减少第一圈螺纹牙受力\hint{可采用悬置螺母、内斜螺母、
        环槽螺母,使螺栓螺母变形同步,或平均各段载荷}。
        \item 减小螺栓应力幅\hint{降低螺栓刚度$C_1$,如采用腰状杆螺栓、增大螺栓长度、
        加弹性元件;或提高被连接件刚度$C_2$,如用刚性密封元件替代弹性密封元件}。
        \item 减少应力集中\hint{增大牙根圆角半径,螺纹收尾处设退刀槽等}。
        \item 避免附加完全应力\hint{主要是避免偏心拉}。
        \item 采用合理的制造工艺\hint{特殊工艺处理}。
    \end{enumerate}
\end{itemize}

\section{轴与轴系}
\begin{itemize}\tightlist
    \item 基本作用:支撑回转零件,传递运动与转矩。
    \item 按承载分类:
    \begin{itemize}\tightlist
        \item 心轴:只承受弯矩\hint{不限制轴向转动或不转动,如滑轮主轴}。
        \item 传动轴:只承受转矩\hint{不约束横向转动或不弯曲,如汽车发动机至后桥的轴}。
        \item 转轴:既承受弯矩又承受转矩\hint{对弯曲与转动均有约束,最常见}。
    \end{itemize}
    \item 按形状分类:直轴\hint{光轴及阶梯轴}\minor{、曲轴、挠性轴}。
\end{itemize}

\subsection{轴的强度、刚度、振动强度计算}
\begin{itemize}\tightlist
    \item 轴的失效形式:疲劳断裂、过载断裂、刚度不足\minor{、振幅太大、磨损、蠕变}。
    \item 强度计算:初选轴径或初步校核\hint{可按纯扭转强度或弯扭合成强度计算}。
    \begin{itemize}\tightlist
        \item 按纯扭转强度计算:
        \begin{gather}
        \tau=\frac{T}{W_p}\approx\frac{9550\times10^3\frac Pn}{0.2d^3}\leq[\tau]\\
        d\geq\sqrt[3]{\frac{9550\times10^3}{0.2\times[\tau]}\cdot\frac Pn}=A\times
        \sqrt[3]{\frac Pn}
        \end{gather}
        其中$A$为计算系数\hint{计入了弯矩影响},可根据轴的材料查表。
        \item 按弯扭合成强度计算:作计算简图\hint{轴承为支点},分别作弯矩与转矩图,按第三
        强度理论\footnote{公式中,$\alpha$为折合系数\hint{考虑循环变化},$\delta$为实际
        的循环特征值;对称循环下$\delta=-1$,$\alpha=1$;脉动循环下$\delta=0$,$\alpha
        \approx0.6$;静应力下$\delta=1$,$\alpha\approx0.3$。}
        \begin{equation}
        M_v=\sqrt{M^2+(\alpha T)^2},\quad\alpha=\frac{[\sigma_{-1}]}{[\sigma_\delta]}
        \end{equation}
        合成当量弯矩,再按$\sigma_v=\frac{M_v}{W_z}\leq[\sigma]$校核,此处$[\sigma]$
        在转轴与转动心轴上取$[\sigma_{-1}]$,对其他变化较小情况取$[\sigma_0]$。
        \item \emph{轴上开一个键槽,轴径应增大$3\%\sim5\%$;两个键槽,增大
        $7\%\sim10\%$。}
    \end{itemize}
    \item 刚度校核:
    \begin{itemize}\tightlist
        \item \minor{弯曲刚度:$y\leq[y]$及$\theta\leq[\theta]$\hint{用能量法或当量
        直径计算}。}
        \item 扭转刚度:$\varphi=\frac{584}G\sum_{i=1}^n\dfrac{T_il_i}{d_i^4}\leq
        [\varphi]$\minor{;若含键槽,应乘以刚度降低\hint{变形增大}系数
        $k=\frac1{1-4fh/d}$}。
    \end{itemize}
    \item (弯曲)振动计算:主要考虑一阶临界转速$n_{e1}$,刚性轴要求$n<0.76n_{e1}$。
\end{itemize}

\subsection{轴系设计}
\begin{itemize}\tightlist
    \item 设计要素:材料选择\hint{合用且经济}、结构设计\hint{轴系改错}、工作能力校核。
    \hint{强度、刚度、振动强度}。
    \item 常用材料:碳钢\hint{一般常用}、合金钢\hint{性能好、昂贵,应热处理}、球墨铸铁
    及高强度铸铁\hint{耐磨吸振、易加工、便宜,用于加工形状特殊的轴}\minor{;查表选择}。
    \item 结构设计要素:
    \begin{enumerate}\tightlist
        \item 受力合理、平均,有利于提高强度及刚度。
        \item 定位准确,固定可靠\hint{常见错误}。
        \item 便于加工、装拆、调整\hint{常见错误}。
        \item 减少应力集中,节省材料、减轻质量。
    \end{enumerate}
    \item 轴端划分:轴颈\hint{与轴承配合}、轴头\hint{与齿轮、带轮等一般零件配合}、
    轴肩/轴环\hint{轴向定位用}。
    \item 轴系改错基本原则:固定、定位、可装拆、可靠。
    \item \emph{轴系改错十二条:}
    \begin{enumerate}\tightlist
        \item 轴上各零件应完成周向或轴向固定,不能有多余自由度\hint{常见错误:轴端的
        联轴器、带轮等零件未固定,缺键或轴端挡圈;轴端挡圈或压板直径太小}。
        \item 阶梯轴的阶梯应当是递增或递减
        \footnote{即使相等也不行——特别是有螺纹段与无螺纹段间。}
        的,以便装卸\hint{常见错误:有凸出的轴环,或有下陷段}。
        \item 轴颈、轴头的长度应略小于与之配合的零件长度\hint{可用退刀槽、砂轮越程槽
        代替},以便可靠地固定\hint{否则无法接触},并利于拆卸。
        \item 固定轴承之轴肩、轴环、套筒、圆螺母的外径应低于轴承内圈,以便用拉马拆卸。
        \item 轴上键槽应打通,使得轴可以自由进出\footnote{键只负责周向定位,在轴向上
        不应受到约束。}。
        \item 正确选择轴承\hint{常见错误:选用单个圆锥、圆柱滚子轴承,不能承受轴向移动;
        成对的轴承方向没有形成O型或X型的对称结构}。
        \item 轴承座与端盖配合处应加工凸台,需要加密封垫圈\hint{用粗黑线表示},并需要
        与轴承外圈配合\footnote{若轴采用一端双向固定,一端游动的结构\hint{如在一侧加用
        套杯},则游动的端盖一侧不应与轴承接触。}。
        \item 轴承配合处不需要键\hint{一般用肩、环等固定,或采用过盈配合}。
        \item 键槽不应太长,以免键的空间位置与其他轴上零件冲突。
        \item 轴承座剖面线应呈上下箱体结构\hint{下面和上面各一系列},以保证同轴度
        \hint{上下拼合}。
        \item 锥面\hint{常来自于齿轮、蜗轮等}不能实现轴向定位。
        \item 多个平键应位于同一母线上\hint{减少装夹时间}。
    \end{enumerate}
    \item 固定件的选择:
    \begin{itemize}\tightlist
        \item 轴端长度:轴头长度比零件轮毂宽度小$1\sim\SI{2}{mm}$;轴颈长度一般等于
        轴承宽度;轴肩或轴环高度$h$取约$0.1d$,圆角半径$r\approx0.75d$,长度$b
        \approx1.4h$。
        \item 轴向固定:轴肩或轴环\hint{简单可靠,应力集中}、套筒\hint{简单可靠,不宜
        过长,可与轴肩、环配合固定两个零件}、圆螺母\hint{装拆方便,应力集中,一般用于
        轴端零件,应采用双螺母或止动垫圈结构}、弹性挡圈\hint{结构简单,应力集中,一般用
        于轴承,常与轴肩配合}、轴端压板\hint{固定轴端零件,需要防转}。
        \item 周向固定:键\hint{(普通)平键、半圆键、花键,需校核挤压强度}、紧定螺钉
        或销钉\hint{常用于光轴,双向固定,可靠性低}、过盈配合\hint{重压需用温差法或压力
        机装配,可靠性高,对工艺要求高,装拆不便}。
        \item 轴对机架定位:由轴承实现。
    \end{itemize}
    \item 结构工艺性:结构尽可能简单\hint{如阶梯少、圆角倒角等尺寸统一},螺纹段留出对应
    的槽,加工精度与表面粗糙度合理选择\hint{不宜定得过高}。
    \item 提高强度与刚度的措施:
    \begin{enumerate}\tightlist
        \item 合理布置零件,平均分载:输入零件应布置在轴中部或轴承附近;减少悬臂长度
        \minor{;固结多余的转轴为心轴}。
        \item 改进轴的结构,降低应力集中:用大圆角\minor{、加中心环、开卸荷槽,少开
        螺纹}。
        \item 改善轴的表面质量:减小粗糙度,表面处理。
    \end{enumerate}
    \item \minor{工作能力校核:安全系数法\hint{查表叠加}。}
\end{itemize}

\section{轴承}
\begin{itemize}\tightlist
    \item 轴承的功用:支撑轴系,减小轴与支承间的摩擦、磨损。
    \item 两类轴承:滑动轴承\hint{抗冲击、高精度,用于低速重载、精密仪器、一般支承场合}
    ,滚动轴承\hint{摩擦小、启动快、效率高、已标准化,用于一般机器}。
\end{itemize}

\subsection{滑动轴承}
\begin{itemize}\tightlist
    \item 分类:径向、止推、径向止推;液体润滑\hint{无直接接触}、非液体润滑\hint{有
    微观上直接接触};液体、气体、固体。
    \item 结构组成:轴承座/轴承盖、轴套/轴瓦\hint{与轴接触}、油孔、油槽、紧固件。
    \item 结构形式:整体式\hint{成本低;装拆不便、磨损后不可调隙}、对开式\hint{易调整,
    应用广泛}、自动调心式\hint{补偿径向变形}。
    \item 轴瓦设计\hint{常在表面用轴承合金加工出轴承衬};轴瓦结构形式:整体式、对开式;
    厚壁、薄壁;轴瓦定位:销钉定位、止动螺钉定位、凸缘定位、凸耳定位。
    \item 油孔油槽设计原则:开在油膜压力最小处\hint{一般远离承载区}或轴承剖分面处;油槽
    不能沿轴向完全开通\hint{避免漏油};不允许在轴瓦承载区开槽\hint{不能降低轴承强度和
    油膜承载能力}。
    \item 滑动轴承失效形式:磨粒磨损、粘着磨损\hint{胶合}、疲劳磨损、腐蚀。
    \item 轴承材料选择原则:减摩性、耐磨性、抗胶合性;嵌藏性\hint{容纳磨粒}、顺应性
    \hint{通过弹性变形补偿偏斜}、磨合性;强度、抗腐蚀性;导热性、工艺性、经济性。
    \item 常用轴承材料:轴承合金\hint{硬晶粒+软基体,各方面性能优良,疲劳强度低}、
    铜合金\hint{强度和减摩、耐磨性好,磨合性及嵌藏性差,锡、铝、铅、黄四种各有所长}、
    铸铁\hint{经济,有一定润滑作用,磨合性差,只适用于轻载低速}、多孔金属\hint{自带
    润滑油,但不耐冲击,适用于载荷平稳、加油不便场合}、非金属材料\hint{塑料性能较好,
    不耐热,用于常温轻载;橡胶用于水润场合}。
    \item 工作能力计算:已知$d,n,F_r$,先根据工况确定结构形式与材料,再分别验算:
    \begin{itemize}\tightlist
        \item 平均压力$p=\frac{F_r}{dB}\leq[p]$\hint{抗挤压强度};
        \item 热载荷$pv=\leq[pv]$\hint{防胶合};
        \item 滑动速度$v\leq[v]$\hint{防止局部$p$及$pv$过大}。
    \end{itemize}
    对止推轴承,也需验算$p$及$pv$,但截面积$A$与径向轴承不同。
    \item \emph{动压润滑条件:}
    \begin{enumerate}\tightlist
        \item 形成收敛楔形间隙\hint{必须有偏心距$e$};
        \item 接触面间充满粘性液体;
        \item 接触面间有相对滑动\hint{必须在一定转速下才能发生润滑}。
    \end{enumerate}
    \item 动压润滑与静压润滑区别:初压有无,是否需要供油设备。
    \item \minor{其他类型轴承:动静压润滑轴承、气体轴承、磁力轴承。}
\end{itemize}

\subsection{滚动轴承}
\begin{itemize}\tightlist
    \item 结构:外圈、内圈、滚动体\hint{球或滚子}、保持架。
    \item 分类:按公称接触角$\alpha$,分为向心轴承\hint{$\alpha\leq\SI{45}{\degree%
    }$,承受径向载荷}与推力轴承\hint{$\alpha>\SI{45}{\degree}$,承受轴向载荷}。
    \item 主要性能:承载能力\hint{滚子优于球$1.5\sim3$倍}、极限转速、允许角偏差\hint%
    {能否调心}。
    \item 滚动轴承类别:
    \begin{itemize}\tightlist
        \item 调心球轴承\hint{1}:径向为主,可调心;
        \item 调心滚子轴承\hint{2}:径向为主,可调心,可承受重载;
        \item 圆锥滚子轴承\hint{3}:径向轴向皆可承受,外圈可分离,成对使用;
        \item 推力球轴承\hint{5}:只受轴向,不耐高速;
        \item 深沟球轴承\hint{6}:径向为主,最常用,价格最低;
        \item 角接触球轴承\hint{7}:径向轴向皆可承受,$\alpha$可取\SI{15}{\degree}
        \hint{A}、\SI{25}{\degree}\hint{AC}、\SI{40}{\degree}\hint{B}三种,成对
        使用;
        \item 圆柱滚子轴承\hint{N}:只受径向,外圈可分离,不可调心,耐冲击。
    \end{itemize}
    \item 轴承(基本)代号:类型代号-[宽度系列代号-]直径系列代号-内径代号\minor{%
    [-内部结构代号-公差等级代号-游隙代号]}
    \begin{itemize}\tightlist
        \item 类型代号:见“滚动轴承类别”。
        \item 宽度系列代号:\minor{系列为$8012345$,}一般为$0$,除调心轴承与圆锥滚子
        轴承外均可省略。
        \item 直径系列代号:\minor{系列为$789012345$,}不能省略。
        \item 内径代号:$00-03$分别对应$10,12,15,\SI{17}{mm}$,$04-99$乘五即得对应
        直径\hint{mm},超过$\SI{500}{mm}$的用分母形式表示。
        \item 内部结构代号:如角接触轴承的三种$\alpha$。
        \item 公差等级代号:\minor{系列为$0(6x)6542$,}一般为$0$并省略\minor{,否则
        应标出``/Pn''}。
        \item 游隙代号:\minor{系列为$120345$},一般为$0$并省略\minor{,否则应标出
        ``/Cn''}。
    \end{itemize}
    \item 滚动轴承的选择依据:
    \begin{enumerate}\tightlist
        \item 轻载或高速宜用球轴承,冲击或重载宜用滚子轴承;能用球轴承时不要用滚子轴承
        \hint{经济性}。
        \item 纯径向载荷宜用向心轴承,纯轴向宜用推力轴承;同时承受径向与轴向载荷,需结
        合两者相对大小具体考虑。
        \item 考虑调心需求\hint{此时避免用圆柱滚子轴承}。
        \item 整体轴承座中的轴承,优先考虑外圈可分离的\hint{便于拆卸}。
    \end{enumerate}
    \item 滚动轴承的载荷情况:
    \begin{enumerate}\tightlist
        \item 理论承载区为整个下半部分,实际承载区小于一般;附加轴向载荷有助于扩大承
        载区\hint{预紧}。
        \item $\alpha>0$时,总有轴向附加力$S=F_R\tan\alpha$\hint{应成对使用}。
        \item 元件所受载荷总呈脉动循环:外圈是稳定脉动循环,内圈不稳定,滚动体上循环
        频率最高\hint{公转+自转}。
    \end{enumerate}
    \item 常见失效形式:疲劳点蚀\hint{主要,脉动循环导致};磨粒磨损、屈服失效;轴承故障
    \hint{人为因素导致}。
    \item 寿命校核基本概念:基本额定寿命$L_{10}$\hint{确定条件下,失效率达10\%}、基本
    额定寿命载荷$C$\hint{确定条件下使$L_{10}$达$10^6$转}、基本额定静载荷$C_0$\hint{确定条件下使接触应力达一定值}。
    \item 动载寿命校核:用$P=(XF_r+YF_a)f_p$折算实际情况为当量载荷$P$,用公式
    \footnote{公式中,$\varepsilon$为寿命指数,对球轴承取$3$,滚子轴承取$10/3$。}
    \begin{equation}
        L_{10}=\left(\frac CP\right)^\varepsilon\times10^6
    \end{equation}
    核算寿命\hint{为多少转,可进一步换算为小时数}。
    \item \minor{可查表将已算好的$L_{10}$转化为其他的$L_n$。}
    \item \minor{静载校核条件:$C_0\geq S_0P$,$S_0$查表,$P$需根据轴承类型合成、折
    算。}
    \item 角接触球轴承的轴向载荷计算:
    \begin{itemize}\tightlist
        \item 装法:正装\hint{X型,接触法线向内,轴向附加力对内},反装\hint{O型,
        接触法线向外,轴向附加力对外}。
        \item 计算原理:先做受力简图,理论上应有
        \begin{equation}\label{eq:轴向平衡条件}
        F_A+S_1=S_2
        \end{equation}
        实际上$F_a$往往偏大或偏小,此时应在
        方程一端附加支反力$S_1'$或$S_2'$\hint{来自于支承接触面}以维持等号。
        \item \emph{$S_1$与$S_2$由径向载荷决定,不能改变,可变的是附加支反力!}
        \item 做校核时,实际轴向力$F_a$取方程(\ref{eq:轴向平衡条件})两侧之最大值。
    \end{itemize}
    \item 轴承支承方式:两端单向固定;一段双向固定、一端游动\hint{留出自由伸缩余地};
    两端游动\hint{便于调整,如自适应的人字齿}。
    \item 利用调整垫片可控制游隙。
    \item \minor{轴承固定:内圈——轴肩、弹性挡圈、轴端压板、圆螺母-止动垫圈、锥形套筒;
    外圈——孔用挡肩、轴承端盖、弹性挡圈、外圈带止动槽-止动环。}
    \item \minor{轴承是专门生产的标准件,不允许加工。}
    \item 轴承装拆原则:不允许通过滚动体传力\hint{不能使内外圈错位}。
    \item 轴承预紧:针对成对轴承较为有效;分定位预紧\hint{“形锁合”}与定压预紧\hint{%
    “力锁合”}。
    \item 润滑与密封:油润滑与脂润滑;\minor{润滑方式包括油浴、滴油、飞溅、喷油、油雾、
    油气等;}密封分接触式\hint{毡圈、唇型密封圈}与非接触式\hint{间隙式、迷宫式}。
\end{itemize} 

\section{联轴器与离合器}
\begin{itemize}\tightlist
    \item 共性:连接两轴及轴上回转件,传递运动和转矩。。
    \item \emph{区别:联轴器只能在停车时才能拆开;离合器可在工作中接合或分离。}
    \item 联轴器的特殊要求:补偿轴的偏移\hint{能补偿者为挠性,否则为刚性};吸振缓冲。
    \item 刚性联轴器:套筒联轴器\hint{用锥销或平键连接},凸缘联轴器\hint{配合对中,或
    用受剪螺钉固定}。
    \item 挠性联轴器:分无弹性元件\hint{利用可动元件配合}和有弹性元件两种。
    \begin{itemize}\tightlist
        \item 无弹性元件:十字滑块联轴器、齿轮联轴器\hint{耐重载}、万向联轴器
        \hint{轴线相交}。
        \item 有弹性元件\hint{高速、冲击适用}:弹性套柱销联轴器、弹性柱销联轴器。
    \end{itemize}
    \item 联轴器的选择:先初选类别,查手册,校核转矩$T_c=KT\leq[T]$\hint{含工况系数}
    及转速$n\leq[n]$。
    \item 离合器类别:
    \begin{itemize}\tightlist
        \item 嵌合式离合器:同步回转,工作可靠,有刚冲。实例:牙嵌式、齿轮式。
        \item 摩擦式离合器:无级变速,工作平稳,不同步。实例:单盘式、多盘式。
        \item \minor{超越式离合器:单方向传矩。实例:滚珠式、棘轮式\hint{见“棘轮机构”
        章节}。}
        \item \minor{安全离合器:转矩过大时分离或打滑,可保护机器。}
    \end{itemize}
\end{itemize}

\end{document}